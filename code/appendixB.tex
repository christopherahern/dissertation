
% Default to the notebook output style

    


% Inherit from the specified cell style.




    
\documentclass{article}

    
    
    \usepackage{graphicx} % Used to insert images
    \usepackage{adjustbox} % Used to constrain images to a maximum size 
    \usepackage{color} % Allow colors to be defined
    \usepackage{enumerate} % Needed for markdown enumerations to work
    \usepackage{geometry} % Used to adjust the document margins
    \usepackage{amsmath} % Equations
    \usepackage{amssymb} % Equations
    \usepackage{eurosym} % defines \euro
    \usepackage[mathletters]{ucs} % Extended unicode (utf-8) support
    \usepackage[utf8x]{inputenc} % Allow utf-8 characters in the tex document
    \usepackage{fancyvrb} % verbatim replacement that allows latex
    \usepackage{grffile} % extends the file name processing of package graphics 
                         % to support a larger range 
    % The hyperref package gives us a pdf with properly built
    % internal navigation ('pdf bookmarks' for the table of contents,
    % internal cross-reference links, web links for URLs, etc.)
    \usepackage{hyperref}
    \usepackage{longtable} % longtable support required by pandoc >1.10
    \usepackage{booktabs}  % table support for pandoc > 1.12.2
    

    
    
    \definecolor{orange}{cmyk}{0,0.4,0.8,0.2}
    \definecolor{darkorange}{rgb}{.71,0.21,0.01}
    \definecolor{darkgreen}{rgb}{.12,.54,.11}
    \definecolor{myteal}{rgb}{.26, .44, .56}
    \definecolor{gray}{gray}{0.45}
    \definecolor{lightgray}{gray}{.95}
    \definecolor{mediumgray}{gray}{.8}
    \definecolor{inputbackground}{rgb}{.95, .95, .85}
    \definecolor{outputbackground}{rgb}{.95, .95, .95}
    \definecolor{traceback}{rgb}{1, .95, .95}
    % ansi colors
    \definecolor{red}{rgb}{.6,0,0}
    \definecolor{green}{rgb}{0,.65,0}
    \definecolor{brown}{rgb}{0.6,0.6,0}
    \definecolor{blue}{rgb}{0,.145,.698}
    \definecolor{purple}{rgb}{.698,.145,.698}
    \definecolor{cyan}{rgb}{0,.698,.698}
    \definecolor{lightgray}{gray}{0.5}
    
    % bright ansi colors
    \definecolor{darkgray}{gray}{0.25}
    \definecolor{lightred}{rgb}{1.0,0.39,0.28}
    \definecolor{lightgreen}{rgb}{0.48,0.99,0.0}
    \definecolor{lightblue}{rgb}{0.53,0.81,0.92}
    \definecolor{lightpurple}{rgb}{0.87,0.63,0.87}
    \definecolor{lightcyan}{rgb}{0.5,1.0,0.83}
    
    % commands and environments needed by pandoc snippets
    % extracted from the output of `pandoc -s`
    \DefineVerbatimEnvironment{Highlighting}{Verbatim}{commandchars=\\\{\}}
    % Add ',fontsize=\small' for more characters per line
    \newenvironment{Shaded}{}{}
    \newcommand{\KeywordTok}[1]{\textcolor[rgb]{0.00,0.44,0.13}{\textbf{{#1}}}}
    \newcommand{\DataTypeTok}[1]{\textcolor[rgb]{0.56,0.13,0.00}{{#1}}}
    \newcommand{\DecValTok}[1]{\textcolor[rgb]{0.25,0.63,0.44}{{#1}}}
    \newcommand{\BaseNTok}[1]{\textcolor[rgb]{0.25,0.63,0.44}{{#1}}}
    \newcommand{\FloatTok}[1]{\textcolor[rgb]{0.25,0.63,0.44}{{#1}}}
    \newcommand{\CharTok}[1]{\textcolor[rgb]{0.25,0.44,0.63}{{#1}}}
    \newcommand{\StringTok}[1]{\textcolor[rgb]{0.25,0.44,0.63}{{#1}}}
    \newcommand{\CommentTok}[1]{\textcolor[rgb]{0.38,0.63,0.69}{\textit{{#1}}}}
    \newcommand{\OtherTok}[1]{\textcolor[rgb]{0.00,0.44,0.13}{{#1}}}
    \newcommand{\AlertTok}[1]{\textcolor[rgb]{1.00,0.00,0.00}{\textbf{{#1}}}}
    \newcommand{\FunctionTok}[1]{\textcolor[rgb]{0.02,0.16,0.49}{{#1}}}
    \newcommand{\RegionMarkerTok}[1]{{#1}}
    \newcommand{\ErrorTok}[1]{\textcolor[rgb]{1.00,0.00,0.00}{\textbf{{#1}}}}
    \newcommand{\NormalTok}[1]{{#1}}
    
    % Define a nice break command that doesn't care if a line doesn't already
    % exist.
    \def\br{\hspace*{\fill} \\* }
    % Math Jax compatability definitions
    \def\gt{>}
    \def\lt{<}
    % Document parameters
    \title{appendixB}
    
    
    

    % Pygments definitions
    
\makeatletter
\def\PY@reset{\let\PY@it=\relax \let\PY@bf=\relax%
    \let\PY@ul=\relax \let\PY@tc=\relax%
    \let\PY@bc=\relax \let\PY@ff=\relax}
\def\PY@tok#1{\csname PY@tok@#1\endcsname}
\def\PY@toks#1+{\ifx\relax#1\empty\else%
    \PY@tok{#1}\expandafter\PY@toks\fi}
\def\PY@do#1{\PY@bc{\PY@tc{\PY@ul{%
    \PY@it{\PY@bf{\PY@ff{#1}}}}}}}
\def\PY#1#2{\PY@reset\PY@toks#1+\relax+\PY@do{#2}}

\expandafter\def\csname PY@tok@gd\endcsname{\def\PY@tc##1{\textcolor[rgb]{0.63,0.00,0.00}{##1}}}
\expandafter\def\csname PY@tok@gu\endcsname{\let\PY@bf=\textbf\def\PY@tc##1{\textcolor[rgb]{0.50,0.00,0.50}{##1}}}
\expandafter\def\csname PY@tok@gt\endcsname{\def\PY@tc##1{\textcolor[rgb]{0.00,0.27,0.87}{##1}}}
\expandafter\def\csname PY@tok@gs\endcsname{\let\PY@bf=\textbf}
\expandafter\def\csname PY@tok@gr\endcsname{\def\PY@tc##1{\textcolor[rgb]{1.00,0.00,0.00}{##1}}}
\expandafter\def\csname PY@tok@cm\endcsname{\let\PY@it=\textit\def\PY@tc##1{\textcolor[rgb]{0.25,0.50,0.50}{##1}}}
\expandafter\def\csname PY@tok@vg\endcsname{\def\PY@tc##1{\textcolor[rgb]{0.10,0.09,0.49}{##1}}}
\expandafter\def\csname PY@tok@m\endcsname{\def\PY@tc##1{\textcolor[rgb]{0.40,0.40,0.40}{##1}}}
\expandafter\def\csname PY@tok@mh\endcsname{\def\PY@tc##1{\textcolor[rgb]{0.40,0.40,0.40}{##1}}}
\expandafter\def\csname PY@tok@go\endcsname{\def\PY@tc##1{\textcolor[rgb]{0.53,0.53,0.53}{##1}}}
\expandafter\def\csname PY@tok@ge\endcsname{\let\PY@it=\textit}
\expandafter\def\csname PY@tok@vc\endcsname{\def\PY@tc##1{\textcolor[rgb]{0.10,0.09,0.49}{##1}}}
\expandafter\def\csname PY@tok@il\endcsname{\def\PY@tc##1{\textcolor[rgb]{0.40,0.40,0.40}{##1}}}
\expandafter\def\csname PY@tok@cs\endcsname{\let\PY@it=\textit\def\PY@tc##1{\textcolor[rgb]{0.25,0.50,0.50}{##1}}}
\expandafter\def\csname PY@tok@cp\endcsname{\def\PY@tc##1{\textcolor[rgb]{0.74,0.48,0.00}{##1}}}
\expandafter\def\csname PY@tok@gi\endcsname{\def\PY@tc##1{\textcolor[rgb]{0.00,0.63,0.00}{##1}}}
\expandafter\def\csname PY@tok@gh\endcsname{\let\PY@bf=\textbf\def\PY@tc##1{\textcolor[rgb]{0.00,0.00,0.50}{##1}}}
\expandafter\def\csname PY@tok@ni\endcsname{\let\PY@bf=\textbf\def\PY@tc##1{\textcolor[rgb]{0.60,0.60,0.60}{##1}}}
\expandafter\def\csname PY@tok@nl\endcsname{\def\PY@tc##1{\textcolor[rgb]{0.63,0.63,0.00}{##1}}}
\expandafter\def\csname PY@tok@nn\endcsname{\let\PY@bf=\textbf\def\PY@tc##1{\textcolor[rgb]{0.00,0.00,1.00}{##1}}}
\expandafter\def\csname PY@tok@no\endcsname{\def\PY@tc##1{\textcolor[rgb]{0.53,0.00,0.00}{##1}}}
\expandafter\def\csname PY@tok@na\endcsname{\def\PY@tc##1{\textcolor[rgb]{0.49,0.56,0.16}{##1}}}
\expandafter\def\csname PY@tok@nb\endcsname{\def\PY@tc##1{\textcolor[rgb]{0.00,0.50,0.00}{##1}}}
\expandafter\def\csname PY@tok@nc\endcsname{\let\PY@bf=\textbf\def\PY@tc##1{\textcolor[rgb]{0.00,0.00,1.00}{##1}}}
\expandafter\def\csname PY@tok@nd\endcsname{\def\PY@tc##1{\textcolor[rgb]{0.67,0.13,1.00}{##1}}}
\expandafter\def\csname PY@tok@ne\endcsname{\let\PY@bf=\textbf\def\PY@tc##1{\textcolor[rgb]{0.82,0.25,0.23}{##1}}}
\expandafter\def\csname PY@tok@nf\endcsname{\def\PY@tc##1{\textcolor[rgb]{0.00,0.00,1.00}{##1}}}
\expandafter\def\csname PY@tok@si\endcsname{\let\PY@bf=\textbf\def\PY@tc##1{\textcolor[rgb]{0.73,0.40,0.53}{##1}}}
\expandafter\def\csname PY@tok@s2\endcsname{\def\PY@tc##1{\textcolor[rgb]{0.73,0.13,0.13}{##1}}}
\expandafter\def\csname PY@tok@vi\endcsname{\def\PY@tc##1{\textcolor[rgb]{0.10,0.09,0.49}{##1}}}
\expandafter\def\csname PY@tok@nt\endcsname{\let\PY@bf=\textbf\def\PY@tc##1{\textcolor[rgb]{0.00,0.50,0.00}{##1}}}
\expandafter\def\csname PY@tok@nv\endcsname{\def\PY@tc##1{\textcolor[rgb]{0.10,0.09,0.49}{##1}}}
\expandafter\def\csname PY@tok@s1\endcsname{\def\PY@tc##1{\textcolor[rgb]{0.73,0.13,0.13}{##1}}}
\expandafter\def\csname PY@tok@kd\endcsname{\let\PY@bf=\textbf\def\PY@tc##1{\textcolor[rgb]{0.00,0.50,0.00}{##1}}}
\expandafter\def\csname PY@tok@sh\endcsname{\def\PY@tc##1{\textcolor[rgb]{0.73,0.13,0.13}{##1}}}
\expandafter\def\csname PY@tok@sc\endcsname{\def\PY@tc##1{\textcolor[rgb]{0.73,0.13,0.13}{##1}}}
\expandafter\def\csname PY@tok@sx\endcsname{\def\PY@tc##1{\textcolor[rgb]{0.00,0.50,0.00}{##1}}}
\expandafter\def\csname PY@tok@bp\endcsname{\def\PY@tc##1{\textcolor[rgb]{0.00,0.50,0.00}{##1}}}
\expandafter\def\csname PY@tok@c1\endcsname{\let\PY@it=\textit\def\PY@tc##1{\textcolor[rgb]{0.25,0.50,0.50}{##1}}}
\expandafter\def\csname PY@tok@kc\endcsname{\let\PY@bf=\textbf\def\PY@tc##1{\textcolor[rgb]{0.00,0.50,0.00}{##1}}}
\expandafter\def\csname PY@tok@c\endcsname{\let\PY@it=\textit\def\PY@tc##1{\textcolor[rgb]{0.25,0.50,0.50}{##1}}}
\expandafter\def\csname PY@tok@mf\endcsname{\def\PY@tc##1{\textcolor[rgb]{0.40,0.40,0.40}{##1}}}
\expandafter\def\csname PY@tok@err\endcsname{\def\PY@bc##1{\setlength{\fboxsep}{0pt}\fcolorbox[rgb]{1.00,0.00,0.00}{1,1,1}{\strut ##1}}}
\expandafter\def\csname PY@tok@mb\endcsname{\def\PY@tc##1{\textcolor[rgb]{0.40,0.40,0.40}{##1}}}
\expandafter\def\csname PY@tok@ss\endcsname{\def\PY@tc##1{\textcolor[rgb]{0.10,0.09,0.49}{##1}}}
\expandafter\def\csname PY@tok@sr\endcsname{\def\PY@tc##1{\textcolor[rgb]{0.73,0.40,0.53}{##1}}}
\expandafter\def\csname PY@tok@mo\endcsname{\def\PY@tc##1{\textcolor[rgb]{0.40,0.40,0.40}{##1}}}
\expandafter\def\csname PY@tok@kn\endcsname{\let\PY@bf=\textbf\def\PY@tc##1{\textcolor[rgb]{0.00,0.50,0.00}{##1}}}
\expandafter\def\csname PY@tok@mi\endcsname{\def\PY@tc##1{\textcolor[rgb]{0.40,0.40,0.40}{##1}}}
\expandafter\def\csname PY@tok@gp\endcsname{\let\PY@bf=\textbf\def\PY@tc##1{\textcolor[rgb]{0.00,0.00,0.50}{##1}}}
\expandafter\def\csname PY@tok@o\endcsname{\def\PY@tc##1{\textcolor[rgb]{0.40,0.40,0.40}{##1}}}
\expandafter\def\csname PY@tok@kr\endcsname{\let\PY@bf=\textbf\def\PY@tc##1{\textcolor[rgb]{0.00,0.50,0.00}{##1}}}
\expandafter\def\csname PY@tok@s\endcsname{\def\PY@tc##1{\textcolor[rgb]{0.73,0.13,0.13}{##1}}}
\expandafter\def\csname PY@tok@kp\endcsname{\def\PY@tc##1{\textcolor[rgb]{0.00,0.50,0.00}{##1}}}
\expandafter\def\csname PY@tok@w\endcsname{\def\PY@tc##1{\textcolor[rgb]{0.73,0.73,0.73}{##1}}}
\expandafter\def\csname PY@tok@kt\endcsname{\def\PY@tc##1{\textcolor[rgb]{0.69,0.00,0.25}{##1}}}
\expandafter\def\csname PY@tok@ow\endcsname{\let\PY@bf=\textbf\def\PY@tc##1{\textcolor[rgb]{0.67,0.13,1.00}{##1}}}
\expandafter\def\csname PY@tok@sb\endcsname{\def\PY@tc##1{\textcolor[rgb]{0.73,0.13,0.13}{##1}}}
\expandafter\def\csname PY@tok@k\endcsname{\let\PY@bf=\textbf\def\PY@tc##1{\textcolor[rgb]{0.00,0.50,0.00}{##1}}}
\expandafter\def\csname PY@tok@se\endcsname{\let\PY@bf=\textbf\def\PY@tc##1{\textcolor[rgb]{0.73,0.40,0.13}{##1}}}
\expandafter\def\csname PY@tok@sd\endcsname{\let\PY@it=\textit\def\PY@tc##1{\textcolor[rgb]{0.73,0.13,0.13}{##1}}}

\def\PYZbs{\char`\\}
\def\PYZus{\char`\_}
\def\PYZob{\char`\{}
\def\PYZcb{\char`\}}
\def\PYZca{\char`\^}
\def\PYZam{\char`\&}
\def\PYZlt{\char`\<}
\def\PYZgt{\char`\>}
\def\PYZsh{\char`\#}
\def\PYZpc{\char`\%}
\def\PYZdl{\char`\$}
\def\PYZhy{\char`\-}
\def\PYZsq{\char`\'}
\def\PYZdq{\char`\"}
\def\PYZti{\char`\~}
% for compatibility with earlier versions
\def\PYZat{@}
\def\PYZlb{[}
\def\PYZrb{]}
\makeatother


    % Exact colors from NB
    \definecolor{incolor}{rgb}{0.0, 0.0, 0.5}
    \definecolor{outcolor}{rgb}{0.545, 0.0, 0.0}



    
    % Prevent overflowing lines due to hard-to-break entities
    \sloppy 
    % Setup hyperref package
    \hypersetup{
      breaklinks=true,  % so long urls are correctly broken across lines
      colorlinks=true,
      urlcolor=blue,
      linkcolor=darkorange,
      citecolor=darkgreen,
      }
    % Slightly bigger margins than the latex defaults
    
    \geometry{verbose,tmargin=1in,bmargin=1in,lmargin=1in,rmargin=1in}
    
    

    \begin{document}
    
    
    \maketitle
    
    

    
    \section{Introduction}\label{introduction}

    Here we define the discrete-time replicator dynamic for signaling games
with arbitrarily many states, messages, and actions. The formulation
provided here treats individual states as independent sender populations
and individual messages as independent receiver populations, instead of
considering the set of all potential sender strategies as a population
and the set of all potential receiver strategies as a population
(cf.~Hofbauer and Huttegger 2015).

By ``devolving'' populations in this manner we dramatically reduce the
dimensions of the system, while also yielding results that are arguably
more transparent. Importantly, it also allows for a fairly
straightforward implementation in algebraic terms which is quick to
compute.

    \subsection{Definitions}\label{definitions}

    We start off by defining the components necessary for our analysis. Once
we have defined these components we can simulate the game dynamics.

    \textbf{First}, we define the payoff matrices for senders and receivers,
which depend solely on the utility functions of senders and receivers
respectively.

$\textbf{A}$ is an $n \times n$ matrix such that
$\textbf{A}_{ij} = U_S(t_i, a_j)$:

\begin{equation}
\textbf{A} =
 \begin{pmatrix}
  U_S(t_1, a_1) & \cdots & U_S(t_1, a_j) & \cdots & U_S(t_1, a_n) \\
  \vdots            & \ddots & \vdots           &           & \vdots \\
  U_S(t_i, a_1) & \cdots & U_S(t_i, a_j) & \cdots & U_S(t_i, a_n) \\  
  \vdots            &  & \vdots           &   \ddots        & \vdots \\
  U_S(t_n, a_1) & \cdots & U_S(t_n, a_j)  & \cdots & U_S(t_n, a_n) \\
 \end{pmatrix}
\end{equation}

$\textbf{B}$ is an $n \times n$ matrix such that
$\textbf{B}_{ij} = U_R(t_i, a_j)$:

\begin{equation}
\textbf{B} =
 \begin{pmatrix}
  U_R(t_1, a_1) & \cdots & U_R(t_1, a_j) & \cdots & U_R(t_1, a_n) \\
  \vdots            & \ddots & \vdots           &           & \vdots \\
  U_R(t_i, a_1) & \cdots & U_R(t_i, a_j) & \cdots & U_R(t_i, a_n) \\  
  \vdots            &  & \vdots           &   \ddots        & \vdots \\
  U_R(t_n, a_1) & \cdots & U_R(t_n, a_j)  & \cdots & U_R(t_n, a_n) \\
 \end{pmatrix}
\end{equation}

    In this case we'll use the standard utility function for a 2x2 Lewis
Signaling Game. Senders and receivers both prefer that the state and
action correspond in some prespecified way. For example, the
correspondence is usally taken as a bijection between states and
actions.

$U_S(t_i,a_j) = U_R(t_i, a_j) = \left\{     \begin{array}{ll}         1  & \mbox{if } i = j \\         0 & \mbox{else}     \end{array} \right. $

We'll define this as a special case of a modified version of the
quadratic loss function used by Crawford and Sobel (1982).

    \begin{Verbatim}[commandchars=\\\{\}]
{\color{incolor}In [{\color{incolor}1}]:} \PY{c}{\PYZsh{} Import numpy}
        \PY{k+kn}{import} \PY{n+nn}{numpy} \PY{k+kn}{as} \PY{n+nn}{np}
        \PY{c}{\PYZsh{} Define the utility functions}
        \PY{k}{def} \PY{n+nf}{U\PYZus{}S}\PY{p}{(}\PY{n}{state}\PY{p}{,} \PY{n}{action}\PY{p}{,} \PY{n}{b}\PY{p}{)}\PY{p}{:}
            \PY{k}{return} \PY{l+m+mi}{1} \PY{o}{\PYZhy{}} \PY{p}{(}\PY{n}{action} \PY{o}{\PYZhy{}} \PY{n}{state} \PY{o}{\PYZhy{}} \PY{p}{(}\PY{l+m+mi}{1}\PY{o}{\PYZhy{}}\PY{n}{state}\PY{p}{)}\PY{o}{*}\PY{n}{b}\PY{p}{)}\PY{o}{*}\PY{o}{*}\PY{l+m+mi}{2}
            \PY{c}{\PYZsh{}return 1 \PYZhy{} abs(action \PYZhy{} state \PYZhy{} (1\PYZhy{}state)*b)}
        \PY{k}{def} \PY{n+nf}{U\PYZus{}R}\PY{p}{(}\PY{n}{state}\PY{p}{,} \PY{n}{action}\PY{p}{)}\PY{p}{:}
            \PY{k}{return} \PY{l+m+mi}{1} \PY{o}{\PYZhy{}} \PY{p}{(}\PY{n}{action} \PY{o}{\PYZhy{}} \PY{n}{state}\PY{p}{)}\PY{o}{*}\PY{o}{*}\PY{l+m+mi}{2}
        \PY{c}{\PYZsh{} Define functions to map integers to interval [0,1]}
        \PY{k}{def} \PY{n+nf}{t}\PY{p}{(}\PY{n}{i}\PY{p}{,} \PY{n}{n}\PY{p}{)}\PY{p}{:}
            \PY{k}{return} \PY{n}{i}\PY{o}{/}\PY{n+nb}{float}\PY{p}{(}\PY{n}{n}\PY{p}{)}
        \PY{k}{def} \PY{n+nf}{a}\PY{p}{(}\PY{n}{i}\PY{p}{,} \PY{n}{n}\PY{p}{)}\PY{p}{:}
            \PY{k}{return} \PY{n}{i}\PY{o}{/}\PY{n+nb}{float}\PY{p}{(}\PY{n}{n}\PY{p}{)}
        \PY{c}{\PYZsh{} Create payoff matrices }
        \PY{k}{print} \PY{l+s}{\PYZdq{}}\PY{l+s}{Sender payoff matrix}\PY{l+s}{\PYZdq{}}
        \PY{n}{A} \PY{o}{=}  \PY{n}{np}\PY{o}{.}\PY{n}{matrix}\PY{p}{(}\PY{p}{[}\PY{p}{[}\PY{n}{U\PYZus{}S}\PY{p}{(}\PY{n}{t}\PY{p}{(}\PY{n}{i}\PY{p}{,} \PY{l+m+mi}{2}\PY{o}{\PYZhy{}}\PY{l+m+mi}{1}\PY{p}{)}\PY{p}{,} \PY{n}{a}\PY{p}{(}\PY{n}{j}\PY{p}{,}\PY{l+m+mi}{2}\PY{o}{\PYZhy{}}\PY{l+m+mi}{1}\PY{p}{)}\PY{p}{,} \PY{l+m+mi}{0}\PY{p}{)} \PY{k}{for} \PY{n}{j} \PY{o+ow}{in} \PY{n+nb}{range}\PY{p}{(}\PY{l+m+mi}{2}\PY{p}{)}\PY{p}{]} \PY{k}{for} \PY{n}{i} \PY{o+ow}{in} \PY{n+nb}{range}\PY{p}{(}\PY{l+m+mi}{2}\PY{p}{)}\PY{p}{]}\PY{p}{)}
        \PY{k}{print} \PY{n}{A}
        \PY{k}{print} \PY{l+s}{\PYZdq{}}\PY{l+s}{Receiver payoff matrix}\PY{l+s}{\PYZdq{}}
        \PY{n}{B} \PY{o}{=} \PY{n}{np}\PY{o}{.}\PY{n}{matrix}\PY{p}{(}\PY{p}{[}\PY{p}{[}\PY{n}{U\PYZus{}R}\PY{p}{(}\PY{n}{t}\PY{p}{(}\PY{n}{i}\PY{p}{,} \PY{l+m+mi}{2}\PY{o}{\PYZhy{}}\PY{l+m+mi}{1}\PY{p}{)}\PY{p}{,} \PY{n}{a}\PY{p}{(}\PY{n}{j}\PY{p}{,}\PY{l+m+mi}{2}\PY{o}{\PYZhy{}}\PY{l+m+mi}{1}\PY{p}{)}\PY{p}{)} \PY{k}{for} \PY{n}{j} \PY{o+ow}{in} \PY{n+nb}{range}\PY{p}{(}\PY{l+m+mi}{2}\PY{p}{)}\PY{p}{]} \PY{k}{for} \PY{n}{i} \PY{o+ow}{in} \PY{n+nb}{range}\PY{p}{(}\PY{l+m+mi}{2}\PY{p}{)}\PY{p}{]}\PY{p}{)}
        \PY{k}{print} \PY{n}{B}
\end{Verbatim}

    \begin{Verbatim}[commandchars=\\\{\}]
Sender payoff matrix
[[ 1.  0.]
 [ 0.  1.]]
Receiver payoff matrix
[[ 1.  0.]
 [ 0.  1.]]
    \end{Verbatim}

    \textbf{Second}, we define the sender and receiver populations.

$\textbf{X}$ is a stochastic population matrix such that the proportion
of the population in $x_i$ using $m_j$ is $x_{ij}$, with
$\sum_j x_{ij} = 1$.

\begin{equation}
\textbf{X} =
 \begin{pmatrix}
  x_{11} &  x_{12} & x_{13} \\
  \vdots        & \vdots & \vdots \\
  x_{i1} &  x_{i2} & x_{i3} \\
  \vdots    & \vdots    & \vdots \\
  x_{n1} &  x_{n2} & x_{n3} \\
 \end{pmatrix}
\end{equation}

Intuitively, each row corresponds to a given state. Each element in the
row corresponds to the proportion of use in that population. Each row
sums to one because the proportion using the various signals must sum to
one.

$\textbf{Y}$ is a population matrix such that the proportion of the
population in $y_i$ responding with action $a_j$ is $y_{ij}$, with
$\sum_j y_{ij} = 1$.

\begin{equation}
\textbf{Y} =
 \begin{pmatrix}
  y_{11} & \cdots & y_{1j}  & \cdots & y_{1n} \\
  y_{21} & \cdots & y_{2j}  & \cdots & y_{2n} \\
  y_{31} & \cdots & y_{3j}  & \cdots & y_{3n} \\
 \end{pmatrix}
\end{equation}

Again, intuitively, each row corresponds to a given message. Each
element in the row corresponds to the proportion of different responses
to the message. Each row sums to one because the proportion using the
various responses must sum to one.

    \begin{Verbatim}[commandchars=\\\{\}]
{\color{incolor}In [{\color{incolor}2}]:} \PY{c}{\PYZsh{} Import random and set seed}
        \PY{k+kn}{import} \PY{n+nn}{random}
        \PY{n}{random}\PY{o}{.}\PY{n}{seed}\PY{p}{(}\PY{l+m+mi}{10}\PY{p}{)}
        \PY{c}{\PYZsh{} Create initial sender matrix}
        \PY{n}{X} \PY{o}{=} \PY{n}{np}\PY{o}{.}\PY{n}{random}\PY{o}{.}\PY{n}{rand}\PY{p}{(}\PY{l+m+mi}{2}\PY{p}{,} \PY{l+m+mi}{2}\PY{p}{)}
        \PY{c}{\PYZsh{} Row\PYZhy{}normalize the sender matrix}
        \PY{n}{X} \PY{o}{/}\PY{o}{=} \PY{n}{X}\PY{o}{.}\PY{n}{sum}\PY{p}{(}\PY{n}{axis}\PY{o}{=}\PY{l+m+mi}{1}\PY{p}{)}\PY{p}{[}\PY{p}{:}\PY{p}{,}\PY{n}{np}\PY{o}{.}\PY{n}{newaxis}\PY{p}{]}
        \PY{k}{print} \PY{n}{X}
        \PY{c}{\PYZsh{} Create initial receiver matrix}
        \PY{n}{Y} \PY{o}{=} \PY{n}{np}\PY{o}{.}\PY{n}{random}\PY{o}{.}\PY{n}{rand}\PY{p}{(}\PY{l+m+mi}{2}\PY{p}{,} \PY{l+m+mi}{2}\PY{p}{)}
        \PY{c}{\PYZsh{} Row\PYZhy{}normalize the receiver matrix}
        \PY{n}{Y} \PY{o}{/}\PY{o}{=} \PY{n}{Y}\PY{o}{.}\PY{n}{sum}\PY{p}{(}\PY{n}{axis}\PY{o}{=}\PY{l+m+mi}{1}\PY{p}{)}\PY{p}{[}\PY{p}{:}\PY{p}{,}\PY{n}{np}\PY{o}{.}\PY{n}{newaxis}\PY{p}{]}
        \PY{k}{print} \PY{n}{Y}
\end{Verbatim}

    \begin{Verbatim}[commandchars=\\\{\}]
[[ 0.4433709   0.5566291 ]
 [ 0.21282727  0.78717273]]
[[ 0.71254386  0.28745614]
 [ 0.78187104  0.21812896]]
    \end{Verbatim}

    $\textbf{P}$ is a stochastic matrix such that
$\forall i \textbf{P}_i = P(t_1),...,P(t_n)$. That is, $\textbf{P}$ is
just $n$ rows of the prior probability distribution over states.

\begin{equation}
\textbf{P} =
 \begin{pmatrix}
  P(t_1) & \cdots & P(t_i)  & \cdots & P(t_n) \\
  \vdots &  & \vdots  & & \vdots  \\
  P(t_1) & \cdots & P(t_i)  & \cdots & P(t_n) \\
 \end{pmatrix}
\end{equation}

    \begin{Verbatim}[commandchars=\\\{\}]
{\color{incolor}In [{\color{incolor}3}]:} \PY{k}{print} \PY{l+s}{\PYZdq{}}\PY{l+s}{Probability matix}\PY{l+s}{\PYZdq{}}
        \PY{n}{P} \PY{o}{=} \PY{n}{np}\PY{o}{.}\PY{n}{matrix}\PY{p}{(}\PY{p}{[}\PY{o}{.}\PY{l+m+mi}{5}\PY{p}{]} \PY{o}{*} \PY{l+m+mi}{4}\PY{p}{)}\PY{o}{.}\PY{n}{reshape}\PY{p}{(}\PY{l+m+mi}{2}\PY{p}{,}\PY{l+m+mi}{2}\PY{p}{)}
        \PY{c}{\PYZsh{}P = np.matrix([[.75, .25], [.75, .25]])}
        \PY{k}{print} \PY{n}{P}
\end{Verbatim}

    \begin{Verbatim}[commandchars=\\\{\}]
Probability matix
[[ 0.5  0.5]
 [ 0.5  0.5]]
    \end{Verbatim}

    For what follows we'll consider the uniform distribution over states,
noting that this is a special case. We can generate a
\href{http://en.wikipedia.org/wiki/Beta-binomial_distribution}{beta-binomial
distribution} over a finite number of states with the following.

    \begin{Verbatim}[commandchars=\\\{\}]
{\color{incolor}In [{\color{incolor}4}]:} \PY{k+kn}{from} \PY{n+nn}{scipy.special} \PY{k+kn}{import} \PY{n}{beta} \PY{k}{as} \PY{n}{beta\PYZus{}func}
        \PY{k+kn}{from} \PY{n+nn}{scipy.misc} \PY{k+kn}{import} \PY{n}{comb}
        \PY{k}{def} \PY{n+nf}{beta\PYZus{}binomial}\PY{p}{(}\PY{n}{n}\PY{p}{,} \PY{n}{alpha}\PY{p}{,} \PY{n}{beta}\PY{p}{)}\PY{p}{:}
            \PY{k}{return} \PY{p}{[}\PY{n}{comb}\PY{p}{(}\PY{n}{n}\PY{o}{\PYZhy{}}\PY{l+m+mi}{1}\PY{p}{,}\PY{n}{k}\PY{p}{)} \PY{o}{*} \PY{n}{beta\PYZus{}func}\PY{p}{(}\PY{n}{k}\PY{o}{+}\PY{n}{alpha}\PY{p}{,} \PY{n}{n}\PY{o}{\PYZhy{}}\PY{l+m+mi}{1}\PY{o}{\PYZhy{}}\PY{n}{k}\PY{o}{+}\PY{n}{beta}\PY{p}{)} \PY{o}{/} \PY{n}{beta\PYZus{}func}\PY{p}{(}\PY{n}{alpha}\PY{p}{,}\PY{n}{beta}\PY{p}{)} \PY{k}{for} \PY{n}{k} \PY{o+ow}{in} \PY{n+nb}{range}\PY{p}{(}\PY{n}{n}\PY{p}{)}\PY{p}{]}
        
        \PY{k}{print} \PY{l+s}{\PYZdq{}}\PY{l+s}{Probability matix}\PY{l+s}{\PYZdq{}}
        \PY{k}{print} \PY{n}{np}\PY{o}{.}\PY{n}{matrix}\PY{p}{(}\PY{p}{(}\PY{n}{beta\PYZus{}binomial}\PY{p}{(}\PY{l+m+mi}{2}\PY{p}{,} \PY{l+m+mi}{1}\PY{p}{,} \PY{l+m+mi}{1}\PY{p}{)} \PY{o}{*} \PY{l+m+mi}{2}\PY{p}{)}\PY{p}{)}\PY{o}{.}\PY{n}{reshape}\PY{p}{(}\PY{l+m+mi}{2}\PY{p}{,}\PY{l+m+mi}{2}\PY{p}{)}
        \PY{k}{print} \PY{l+s}{\PYZdq{}}\PY{l+s}{Another probability matix with uneven weights}\PY{l+s}{\PYZdq{}}
        \PY{k}{print} \PY{n}{np}\PY{o}{.}\PY{n}{matrix}\PY{p}{(}\PY{p}{(}\PY{n}{beta\PYZus{}binomial}\PY{p}{(}\PY{l+m+mi}{2}\PY{p}{,} \PY{l+m+mi}{2}\PY{p}{,} \PY{l+m+mi}{1}\PY{p}{)} \PY{o}{*} \PY{l+m+mi}{2}\PY{p}{)}\PY{p}{)}\PY{o}{.}\PY{n}{reshape}\PY{p}{(}\PY{l+m+mi}{2}\PY{p}{,}\PY{l+m+mi}{2}\PY{p}{)}
\end{Verbatim}

    \begin{Verbatim}[commandchars=\\\{\}]
Probability matix
[[ 0.5  0.5]
 [ 0.5  0.5]]
Another probability matix with uneven weights
[[ 0.33333333  0.66666667]
 [ 0.33333333  0.66666667]]
    \end{Verbatim}

    We can extend this to arbitrarily many states and generate more complex
prior distributions. For example, if we have five states, then we can
visualize the probabilities for different parametrizations. This
includes the uniform distribution as a special case: it's the flat black
line at one fifth, which is indeed flat despite the optical illusion.

    \begin{Verbatim}[commandchars=\\\{\}]
{\color{incolor}In [{\color{incolor}5}]:} \PY{o}{\PYZpc{}} \PY{n}{matplotlib} \PY{n}{inline}
        \PY{k+kn}{from} \PY{n+nn}{matplotlib} \PY{k+kn}{import} \PY{n}{pyplot} \PY{k}{as} \PY{n}{plt}
\end{Verbatim}

    \begin{Verbatim}[commandchars=\\\{\}]
{\color{incolor}In [{\color{incolor}6}]:} \PY{k}{for} \PY{n}{i} \PY{o+ow}{in} \PY{n+nb}{range}\PY{p}{(}\PY{l+m+mi}{5}\PY{p}{)}\PY{p}{:}
            \PY{k}{for} \PY{n}{j} \PY{o+ow}{in} \PY{n+nb}{range}\PY{p}{(}\PY{l+m+mi}{5}\PY{p}{)}\PY{p}{:}
                \PY{n}{plt}\PY{o}{.}\PY{n}{plot}\PY{p}{(}\PY{n}{beta\PYZus{}binomial}\PY{p}{(}\PY{l+m+mi}{5}\PY{p}{,} \PY{n}{j}\PY{p}{,} \PY{n}{i}\PY{p}{)}\PY{p}{)}
        \PY{n}{plt}\PY{o}{.}\PY{n}{axis}\PY{p}{(}\PY{p}{(}\PY{l+m+mi}{0}\PY{p}{,}\PY{l+m+mi}{4}\PY{p}{,}\PY{l+m+mi}{0}\PY{p}{,}\PY{l+m+mi}{1}\PY{p}{)}\PY{p}{)}
        \PY{n}{plt}\PY{o}{.}\PY{n}{show}\PY{p}{(}\PY{p}{)}
\end{Verbatim}

    \begin{Verbatim}[commandchars=\\\{\}]
/home/cahern-adm/anaconda/lib/python2.7/site-packages/IPython/kernel/\_\_main\_\_.py:4: RuntimeWarning: invalid value encountered in double\_scalars
    \end{Verbatim}

    \begin{center}
    \adjustimage{max size={0.9\linewidth}{0.9\paperheight}}{appendixB_files/appendixB_15_1.png}
    \end{center}
    { \hspace*{\fill} \\}
    
    \subsection{Senders}\label{senders}

    \textbf{Third}, now that we have defined these two components, we can
define the expected utility of different strategies and the
discrete-time replicator dynamic. But first we'll remind ourselves of
the initial state of the sender and receiver populations.

    \begin{Verbatim}[commandchars=\\\{\}]
{\color{incolor}In [{\color{incolor}7}]:} \PY{k}{print} \PY{l+s}{\PYZdq{}}\PY{l+s}{Sender matrix}\PY{l+s}{\PYZdq{}}
        \PY{k}{print} \PY{n}{X}
        \PY{k}{print} \PY{l+s}{\PYZdq{}}\PY{l+s}{Receiver matrix}\PY{l+s}{\PYZdq{}}
        \PY{k}{print} \PY{n}{Y}
\end{Verbatim}

    \begin{Verbatim}[commandchars=\\\{\}]
Sender matrix
[[ 0.4433709   0.5566291 ]
 [ 0.21282727  0.78717273]]
Receiver matrix
[[ 0.71254386  0.28745614]
 [ 0.78187104  0.21812896]]
    \end{Verbatim}

    The expected utility of sending message $m_j$ in state $t_i$, where
$\textbf{Y}^T$ is the transpose of $\textbf{Y}$:

\begin{equation}
    E [ x_{ij} ] = (\textbf{A}\textbf{Y}^T)_{ij}
\end{equation}

    \begin{Verbatim}[commandchars=\\\{\}]
{\color{incolor}In [{\color{incolor}8}]:} \PY{k}{print} \PY{l+s}{\PYZdq{}}\PY{l+s}{Sender expected utility}\PY{l+s}{\PYZdq{}}
        \PY{k}{print} \PY{n}{A} \PY{o}{*} \PY{n}{Y}\PY{o}{.}\PY{n}{transpose}\PY{p}{(}\PY{p}{)}
\end{Verbatim}

    \begin{Verbatim}[commandchars=\\\{\}]
Sender expected utility
[[ 0.71254386  0.78187104]
 [ 0.28745614  0.21812896]]
    \end{Verbatim}

    Intuitively, we can read these expected utilities off the receiver
matrix. In fact, the resulting expected utilities are just
$\textbf{Y}^T$. This, however, is particular to the utility functions of
the Lewis signaling game. For any $b > 0$ this does not hold.

    The average expected utility in a sender population $x_i$:

\begin{equation}
    E [ x_{i} ] = (\textbf{X}(\textbf{A}\textbf{Y}^T)^T)_{ii}
\end{equation}

    \begin{Verbatim}[commandchars=\\\{\}]
{\color{incolor}In [{\color{incolor}9}]:} \PY{k}{print} \PY{l+s}{\PYZdq{}}\PY{l+s}{Average sender expected utility}\PY{l+s}{\PYZdq{}}
        \PY{k}{print} \PY{p}{(}\PY{n}{X} \PY{o}{*} \PY{p}{(}\PY{n}{A} \PY{o}{*} \PY{n}{Y}\PY{o}{.}\PY{n}{transpose}\PY{p}{(}\PY{p}{)}\PY{p}{)}\PY{o}{.}\PY{n}{transpose}\PY{p}{(}\PY{p}{)}\PY{p}{)}\PY{o}{.}\PY{n}{diagonal}\PY{p}{(}\PY{p}{)}
\end{Verbatim}

    \begin{Verbatim}[commandchars=\\\{\}]
Average sender expected utility
[[ 0.75113339  0.23288368]]
    \end{Verbatim}

    Note that this average makes sense when we look at the expected
utilities of sending different messages in the different populations.
That is, it is always somewhere in between the two values, which means
things are working as they should.

    Let $\mathbf{\hat{X}}$ be the sender expected utility matrix normalized
by the average expected utilities such that:

\begin{equation}
    \mathbf{\hat{X}}_{ij} = \frac{(\textbf{A}\textbf{Y}^T)_{ij}}{(\textbf{X}(\textbf{A}\textbf{Y}^T)^T)_{ii}}
\end{equation}

    \begin{Verbatim}[commandchars=\\\{\}]
{\color{incolor}In [{\color{incolor}10}]:} \PY{k}{print} \PY{l+s}{\PYZdq{}}\PY{l+s}{Discrete\PYZhy{}time replicator dynamic scaling factors}\PY{l+s}{\PYZdq{}}
         \PY{n}{X\PYZus{}hat} \PY{o}{=} \PY{n}{A} \PY{o}{*} \PY{n}{Y}\PY{o}{.}\PY{n}{transpose}\PY{p}{(}\PY{p}{)} \PY{o}{/} \PY{p}{(}\PY{p}{(}\PY{n}{X} \PY{o}{*} \PY{p}{(}\PY{n}{A} \PY{o}{*} \PY{n}{Y}\PY{o}{.}\PY{n}{transpose}\PY{p}{(}\PY{p}{)}\PY{p}{)}\PY{o}{.}\PY{n}{transpose}\PY{p}{(}\PY{p}{)}\PY{p}{)}\PY{o}{.}\PY{n}{diagonal}\PY{p}{(}\PY{p}{)}\PY{p}{)}\PY{o}{.}\PY{n}{transpose}\PY{p}{(}\PY{p}{)}
         \PY{k}{print} \PY{n}{X\PYZus{}hat}
\end{Verbatim}

    \begin{Verbatim}[commandchars=\\\{\}]
Discrete-time replicator dynamic scaling factors
[[ 0.94862494  1.04092169]
 [ 1.23433355  0.93664342]]
    \end{Verbatim}

    For both sender and receiver populations, under the discrete-time
replicator dynamics strategies grow in proportion to the amount by which
they exceed the average payoff in the population. The discrete-time
replicator dynamic for message $m_j$ in state $t_i$:

\begin{equation}
     x_{ij}' = x_{ij}\frac{E[x_{ij}]}{E[x_i]}
\end{equation}

The sender populations at the next point in time are then given by the
following, where $\otimes$ indicates the element-wise Hadamard product:

\begin{equation}
    \mathbf{X}' = \mathbf{X} \otimes \mathbf{\hat{X}}
\end{equation}

    \begin{Verbatim}[commandchars=\\\{\}]
{\color{incolor}In [{\color{incolor}11}]:} \PY{k}{print} \PY{l+s}{\PYZdq{}}\PY{l+s}{Current sender populations state}\PY{l+s}{\PYZdq{}}
         \PY{k}{print} \PY{n}{X}
         \PY{k}{print} \PY{l+s}{\PYZdq{}}\PY{l+s}{Next sender populations state}\PY{l+s}{\PYZdq{}}
         \PY{n}{X\PYZus{}next} \PY{o}{=} \PY{n}{np}\PY{o}{.}\PY{n}{multiply}\PY{p}{(}\PY{n}{X}\PY{p}{,} \PY{n}{X\PYZus{}hat}\PY{p}{)}
         \PY{k}{print} \PY{n}{X\PYZus{}next}
         \PY{k}{print} \PY{l+s}{\PYZdq{}}\PY{l+s}{Check that sender populations sum to one}\PY{l+s}{\PYZdq{}}
         \PY{k}{print} \PY{n}{np}\PY{o}{.}\PY{n}{sum}\PY{p}{(}\PY{n}{X\PYZus{}next}\PY{p}{,} \PY{n}{axis}\PY{o}{=}\PY{l+m+mi}{1}\PY{p}{)}
\end{Verbatim}

    \begin{Verbatim}[commandchars=\\\{\}]
Current sender populations state
[[ 0.4433709   0.5566291 ]
 [ 0.21282727  0.78717273]]
Next sender populations state
[[ 0.42059269  0.57940731]
 [ 0.26269984  0.73730016]]
Check that sender populations sum to one
[[ 1.]
 [ 1.]]
    \end{Verbatim}

    Now that we have defined the discrete-time replicator dynamics for the
sender populations, we can do the same for the receiver populations with
a few additions. Let $\textbf{C}$ be the conditional probability of a
state given a message. That is, $\textbf{C}_{ij} = P(t_i | m_j)$, where
$\otimes$ indicates element-wise Hadamard multiplication and $\oslash$
indicates the element-wise Hadamard division.

\begin{equation}
\textbf{C} = (P^T \otimes X) \oslash (PX)
\end{equation}

    \begin{Verbatim}[commandchars=\\\{\}]
{\color{incolor}In [{\color{incolor}12}]:} \PY{k}{print} \PY{l+s}{\PYZdq{}}\PY{l+s}{Sender matrix}\PY{l+s}{\PYZdq{}}
         \PY{k}{print} \PY{n}{X}
         \PY{k}{print} \PY{l+s}{\PYZdq{}}\PY{l+s}{Conditional probability of t\PYZus{}i given message m\PYZus{}j}\PY{l+s}{\PYZdq{}}
         \PY{n}{C} \PY{o}{=} \PY{n}{np}\PY{o}{.}\PY{n}{divide}\PY{p}{(}\PY{n}{np}\PY{o}{.}\PY{n}{multiply}\PY{p}{(}\PY{n}{P}\PY{o}{.}\PY{n}{transpose}\PY{p}{(}\PY{p}{)}\PY{p}{,} \PY{n}{X}\PY{p}{)}\PY{p}{,} \PY{n}{P} \PY{o}{*} \PY{n}{X}\PY{p}{)}
         \PY{k}{print} \PY{n}{C}
\end{Verbatim}

    \begin{Verbatim}[commandchars=\\\{\}]
Sender matrix
[[ 0.4433709   0.5566291 ]
 [ 0.21282727  0.78717273]]
Conditional probability of t\_i given message m\_j
[[ 0.67566616  0.41421963]
 [ 0.32433384  0.58578037]]
    \end{Verbatim}

    The expected utility of receiver responding to message $m_i$ with action
$a_j$:

\begin{equation}
    E [ y_{ij} ] = (\textbf{B}^T\textbf{C})_{ji}
\end{equation}

Since the resulting matrix is $n \times m$, we swap the indices to get
the appropriate value. Each column corresponds to a receiver population,
and each row corresponds to a response action.

    \begin{Verbatim}[commandchars=\\\{\}]
{\color{incolor}In [{\color{incolor}13}]:} \PY{k}{print} \PY{l+s}{\PYZdq{}}\PY{l+s}{Receiver expected utility}\PY{l+s}{\PYZdq{}}
         \PY{k}{print} \PY{p}{(}\PY{n}{B}\PY{o}{.}\PY{n}{transpose}\PY{p}{(}\PY{p}{)} \PY{o}{*} \PY{n}{C}\PY{p}{)}
         \PY{k}{print} \PY{l+s}{\PYZdq{}}\PY{l+s}{Receiver expected utility of responding to m\PYZus{}i with a\PYZus{}j}\PY{l+s}{\PYZdq{}}
         \PY{k}{print} \PY{p}{(}\PY{n}{B}\PY{o}{.}\PY{n}{transpose}\PY{p}{(}\PY{p}{)} \PY{o}{*} \PY{n}{C}\PY{p}{)}\PY{o}{.}\PY{n}{transpose}\PY{p}{(}\PY{p}{)}
\end{Verbatim}

    \begin{Verbatim}[commandchars=\\\{\}]
Receiver expected utility
[[ 0.67566616  0.41421963]
 [ 0.32433384  0.58578037]]
Receiver expected utility of responding to m\_i with a\_j
[[ 0.67566616  0.32433384]
 [ 0.41421963  0.58578037]]
    \end{Verbatim}

    The average expected utility in a receiver population $y_i$:

\begin{equation}
    E [ y_{i} ] = (Y(\textbf{B}^T\textbf{C})_{ji})_{ii}
\end{equation}

    \begin{Verbatim}[commandchars=\\\{\}]
{\color{incolor}In [{\color{incolor}14}]:} \PY{k}{print} \PY{l+s}{\PYZdq{}}\PY{l+s}{Average receiver expected utility}\PY{l+s}{\PYZdq{}}
         \PY{k}{print} \PY{p}{(}\PY{n}{Y} \PY{o}{*} \PY{p}{(}\PY{n}{B}\PY{o}{.}\PY{n}{transpose}\PY{p}{(}\PY{p}{)} \PY{o}{*} \PY{n}{C}\PY{p}{)}\PY{p}{)}\PY{o}{.}\PY{n}{diagonal}\PY{p}{(}\PY{p}{)}
\end{Verbatim}

    \begin{Verbatim}[commandchars=\\\{\}]
Average receiver expected utility
[[ 0.57467353  0.451642  ]]
    \end{Verbatim}

    Let $\mathbf{\hat{Y}}$ be the population normalized matrix for receiver
expected utilities such that:

\begin{equation}
    \mathbf{\hat{Y}}_{ji} = \frac{(\textbf{B}^T\textbf{C})_{ji}}{(\textbf{Y}(\textbf{B}^T\textbf{C})_{ji})_{ii}}
\end{equation}

    \begin{Verbatim}[commandchars=\\\{\}]
{\color{incolor}In [{\color{incolor}15}]:} \PY{k}{print} \PY{l+s}{\PYZdq{}}\PY{l+s}{Discrete\PYZhy{}time replicator dynamic scaling factors}\PY{l+s}{\PYZdq{}}
         \PY{n}{Y\PYZus{}hat} \PY{o}{=} \PY{p}{(}\PY{n}{B}\PY{o}{.}\PY{n}{transpose}\PY{p}{(}\PY{p}{)} \PY{o}{*} \PY{n}{C}\PY{p}{)} \PY{o}{/} \PY{p}{(}\PY{n}{Y} \PY{o}{*} \PY{p}{(}\PY{n}{B}\PY{o}{.}\PY{n}{transpose}\PY{p}{(}\PY{p}{)} \PY{o}{*} \PY{n}{C}\PY{p}{)}\PY{p}{)}\PY{o}{.}\PY{n}{diagonal}\PY{p}{(}\PY{p}{)}
         \PY{k}{print} \PY{n}{Y\PYZus{}hat}\PY{o}{.}\PY{n}{transpose}\PY{p}{(}\PY{p}{)}
\end{Verbatim}

    \begin{Verbatim}[commandchars=\\\{\}]
Discrete-time replicator dynamic scaling factors
[[ 1.17573914  0.5643793 ]
 [ 0.91714153  1.29700153]]
    \end{Verbatim}

    The discrete-time replicator dynamic for action $a_j$ in response to
message $m_i$:

\begin{equation}
     y_{ij}' = y_{ij}\frac{E[y_{ij}]}{E[y_i]}
\end{equation}

The receiver populations at the next point in time are then given by the
following:

\begin{equation}
    \mathbf{Y}' = \mathbf{Y} \otimes \mathbf{\hat{Y}}^T
\end{equation}

    \begin{Verbatim}[commandchars=\\\{\}]
{\color{incolor}In [{\color{incolor}16}]:} \PY{k}{print} \PY{l+s}{\PYZdq{}}\PY{l+s}{Current receiver populations state}\PY{l+s}{\PYZdq{}}
         \PY{k}{print} \PY{n}{Y}
         \PY{k}{print} \PY{l+s}{\PYZdq{}}\PY{l+s}{Next receiver populations state}\PY{l+s}{\PYZdq{}}
         \PY{n}{Y\PYZus{}next} \PY{o}{=} \PY{n}{np}\PY{o}{.}\PY{n}{multiply}\PY{p}{(}\PY{n}{Y}\PY{p}{,} \PY{n}{Y\PYZus{}hat}\PY{o}{.}\PY{n}{transpose}\PY{p}{(}\PY{p}{)}\PY{p}{)}
         \PY{k}{print} \PY{n}{Y\PYZus{}next}
         \PY{k}{print} \PY{l+s}{\PYZdq{}}\PY{l+s}{Check that sender populations sum to one}\PY{l+s}{\PYZdq{}}
         \PY{k}{print} \PY{n}{np}\PY{o}{.}\PY{n}{sum}\PY{p}{(}\PY{n}{Y\PYZus{}next}\PY{p}{,} \PY{n}{axis}\PY{o}{=}\PY{l+m+mi}{1}\PY{p}{)}
\end{Verbatim}

    \begin{Verbatim}[commandchars=\\\{\}]
Current receiver populations state
[[ 0.71254386  0.28745614]
 [ 0.78187104  0.21812896]]
Next receiver populations state
[[ 0.83776571  0.16223429]
 [ 0.7170864   0.2829136 ]]
Check that sender populations sum to one
[[ 1.]
 [ 1.]]
    \end{Verbatim}

    \subsection{Simulations}\label{simulations}

    Now that we have defined the components of the discrete-time replicator
dynamics and how to calculate the state of the populations at the next
point in time, we can calculate the game dynamics from an initial state.

    \begin{Verbatim}[commandchars=\\\{\}]
{\color{incolor}In [{\color{incolor}17}]:} \PY{k}{def} \PY{n+nf}{discrete\PYZus{}time\PYZus{}replicator\PYZus{}dynamics}\PY{p}{(}\PY{n}{n\PYZus{}steps}\PY{p}{,} \PY{n}{X}\PY{p}{,} \PY{n}{Y}\PY{p}{,} \PY{n}{A}\PY{p}{,} \PY{n}{B}\PY{p}{,} \PY{n}{P}\PY{p}{)}\PY{p}{:}
             \PY{l+s+sd}{\PYZdq{}\PYZdq{}\PYZdq{}Calculate the discrete\PYZhy{}time replicator dynamics for\PYZdq{}\PYZdq{}\PYZdq{}}
             \PY{c}{\PYZsh{} Get the number of states, signals, and actions}
             \PY{n}{X\PYZus{}nrow} \PY{o}{=} \PY{n}{X}\PY{o}{.}\PY{n}{shape}\PY{p}{[}\PY{l+m+mi}{0}\PY{p}{]}
             \PY{n}{X\PYZus{}ncol} \PY{o}{=} \PY{n}{X}\PY{o}{.}\PY{n}{shape}\PY{p}{[}\PY{l+m+mi}{1}\PY{p}{]}
             \PY{n}{Y\PYZus{}nrow} \PY{o}{=} \PY{n}{Y}\PY{o}{.}\PY{n}{shape}\PY{p}{[}\PY{l+m+mi}{0}\PY{p}{]} \PY{c}{\PYZsh{} Same as X\PYZus{}ncol}
             \PY{n}{Y\PYZus{}ncol} \PY{o}{=} \PY{n}{Y}\PY{o}{.}\PY{n}{shape}\PY{p}{[}\PY{l+m+mi}{1}\PY{p}{]} \PY{c}{\PYZsh{} Often, but not necessarily, the same as X\PYZus{}nrow}
             \PY{c}{\PYZsh{} Create empty arrays to hold the population states over time}
             \PY{n}{X\PYZus{}t} \PY{o}{=} \PY{n}{np}\PY{o}{.}\PY{n}{empty}\PY{p}{(}\PY{n}{shape}\PY{o}{=}\PY{p}{(}\PY{n}{n\PYZus{}steps}\PY{p}{,} \PY{n}{X\PYZus{}nrow}\PY{o}{*}\PY{n}{X\PYZus{}ncol}\PY{p}{)}\PY{p}{,} \PY{n}{dtype}\PY{o}{=}\PY{n+nb}{float}\PY{p}{)}
             \PY{n}{Y\PYZus{}t} \PY{o}{=} \PY{n}{np}\PY{o}{.}\PY{n}{empty}\PY{p}{(}\PY{n}{shape}\PY{o}{=}\PY{p}{(}\PY{n}{n\PYZus{}steps}\PY{p}{,} \PY{n}{X\PYZus{}nrow}\PY{o}{*}\PY{n}{X\PYZus{}ncol}\PY{p}{)}\PY{p}{,} \PY{n}{dtype}\PY{o}{=}\PY{n+nb}{float}\PY{p}{)}
             \PY{c}{\PYZsh{} Set the initial state}
             \PY{n}{X\PYZus{}t}\PY{p}{[}\PY{l+m+mi}{0}\PY{p}{,}\PY{p}{:}\PY{p}{]} \PY{o}{=} \PY{n}{X}\PY{o}{.}\PY{n}{ravel}\PY{p}{(}\PY{p}{)}
             \PY{n}{Y\PYZus{}t}\PY{p}{[}\PY{l+m+mi}{0}\PY{p}{,}\PY{p}{:}\PY{p}{]} \PY{o}{=} \PY{n}{Y}\PY{o}{.}\PY{n}{ravel}\PY{p}{(}\PY{p}{)}
             \PY{c}{\PYZsh{} Iterate forward over (n\PYZhy{}1) steps}
             \PY{k}{for} \PY{n}{i} \PY{o+ow}{in} \PY{n+nb}{range}\PY{p}{(}\PY{l+m+mi}{1}\PY{p}{,}\PY{n}{n\PYZus{}steps}\PY{p}{)}\PY{p}{:}
                 \PY{c}{\PYZsh{} Get the previous state}
                 \PY{n}{X\PYZus{}prev} \PY{o}{=} \PY{n}{X\PYZus{}t}\PY{p}{[}\PY{n}{i}\PY{o}{\PYZhy{}}\PY{l+m+mi}{1}\PY{p}{,}\PY{p}{:}\PY{p}{]}\PY{o}{.}\PY{n}{reshape}\PY{p}{(}\PY{n}{X\PYZus{}nrow}\PY{p}{,} \PY{n}{X\PYZus{}ncol}\PY{p}{)}
                 \PY{n}{Y\PYZus{}prev} \PY{o}{=} \PY{n}{Y\PYZus{}t}\PY{p}{[}\PY{n}{i}\PY{o}{\PYZhy{}}\PY{l+m+mi}{1}\PY{p}{,}\PY{p}{:}\PY{p}{]}\PY{o}{.}\PY{n}{reshape}\PY{p}{(}\PY{n}{Y\PYZus{}nrow}\PY{p}{,} \PY{n}{Y\PYZus{}ncol}\PY{p}{)}
                 \PY{c}{\PYZsh{} Calculate the scaling factors}
                 \PY{n}{X\PYZus{}hat} \PY{o}{=} \PY{n}{A} \PY{o}{*} \PY{n}{Y\PYZus{}prev}\PY{o}{.}\PY{n}{transpose}\PY{p}{(}\PY{p}{)} \PY{o}{/} \PY{p}{(}\PY{p}{(}\PY{n}{X\PYZus{}prev} \PY{o}{*} \PY{p}{(}\PY{n}{A} \PY{o}{*} \PY{n}{Y\PYZus{}prev}\PY{o}{.}\PY{n}{transpose}\PY{p}{(}\PY{p}{)}\PY{p}{)}\PY{o}{.}\PY{n}{transpose}\PY{p}{(}\PY{p}{)}\PY{p}{)}\PY{o}{.}\PY{n}{diagonal}\PY{p}{(}\PY{p}{)}\PY{p}{)}\PY{o}{.}\PY{n}{transpose}\PY{p}{(}\PY{p}{)}
                 \PY{n}{C} \PY{o}{=} \PY{n}{np}\PY{o}{.}\PY{n}{divide}\PY{p}{(}\PY{n}{np}\PY{o}{.}\PY{n}{multiply}\PY{p}{(}\PY{n}{P}\PY{o}{.}\PY{n}{transpose}\PY{p}{(}\PY{p}{)}\PY{p}{,} \PY{n}{X\PYZus{}prev}\PY{p}{)}\PY{p}{,} \PY{n}{P} \PY{o}{*} \PY{n}{X\PYZus{}prev}\PY{p}{)}
                 \PY{n}{Y\PYZus{}hat} \PY{o}{=} \PY{p}{(}\PY{n}{B}\PY{o}{.}\PY{n}{transpose}\PY{p}{(}\PY{p}{)} \PY{o}{*} \PY{n}{C}\PY{p}{)} \PY{o}{/} \PY{p}{(}\PY{n}{Y\PYZus{}prev} \PY{o}{*} \PY{p}{(}\PY{n}{B}\PY{o}{.}\PY{n}{transpose}\PY{p}{(}\PY{p}{)} \PY{o}{*} \PY{n}{C}\PY{p}{)}\PY{p}{)}\PY{o}{.}\PY{n}{diagonal}\PY{p}{(}\PY{p}{)}
                 \PY{c}{\PYZsh{}if i == 1:}
                 \PY{c}{\PYZsh{}    print X\PYZus{}hat}
                 \PY{c}{\PYZsh{}    print Y\PYZus{}hat}
                 \PY{c}{\PYZsh{} Calculate next states}
                 \PY{n}{X\PYZus{}t}\PY{p}{[}\PY{n}{i}\PY{p}{,}\PY{p}{:}\PY{p}{]} \PY{o}{=} \PY{n}{np}\PY{o}{.}\PY{n}{multiply}\PY{p}{(}\PY{n}{X\PYZus{}prev}\PY{p}{,} \PY{n}{X\PYZus{}hat}\PY{p}{)}\PY{o}{.}\PY{n}{ravel}\PY{p}{(}\PY{p}{)}
                 \PY{n}{Y\PYZus{}t}\PY{p}{[}\PY{n}{i}\PY{p}{,}\PY{p}{:}\PY{p}{]} \PY{o}{=} \PY{n}{np}\PY{o}{.}\PY{n}{multiply}\PY{p}{(}\PY{n}{Y\PYZus{}prev}\PY{p}{,} \PY{n}{Y\PYZus{}hat}\PY{o}{.}\PY{n}{transpose}\PY{p}{(}\PY{p}{)}\PY{p}{)}\PY{o}{.}\PY{n}{ravel}\PY{p}{(}\PY{p}{)}
             \PY{k}{return} \PY{n}{X\PYZus{}t}\PY{p}{,} \PY{n}{Y\PYZus{}t}
\end{Verbatim}

    We can verify that this simulation matches our calculations above for a
single step of the game dynamics. They do, which suggests we're doing
everything correctly. We can then move on to iterating over a larger
number of time steps and plotting the results.

    \begin{Verbatim}[commandchars=\\\{\}]
{\color{incolor}In [{\color{incolor}18}]:} \PY{n}{X\PYZus{}hist}\PY{p}{,} \PY{n}{Y\PYZus{}hist} \PY{o}{=} \PY{n}{discrete\PYZus{}time\PYZus{}replicator\PYZus{}dynamics}\PY{p}{(}\PY{l+m+mi}{2}\PY{p}{,} \PY{n}{X}\PY{p}{,} \PY{n}{Y}\PY{p}{,} \PY{n}{A}\PY{p}{,} \PY{n}{B}\PY{p}{,} \PY{n}{P}\PY{p}{)}
         \PY{k}{print} \PY{l+s}{\PYZdq{}}\PY{l+s}{Initial sender populations state}\PY{l+s}{\PYZdq{}}
         \PY{k}{print} \PY{n}{X\PYZus{}hist}\PY{p}{[}\PY{l+m+mi}{0}\PY{p}{]}\PY{o}{.}\PY{n}{reshape}\PY{p}{(}\PY{l+m+mi}{2}\PY{p}{,}\PY{l+m+mi}{2}\PY{p}{)}
         \PY{k}{print} \PY{l+s}{\PYZdq{}}\PY{l+s}{Next sender populations state}\PY{l+s}{\PYZdq{}}
         \PY{k}{print} \PY{n}{X\PYZus{}hist}\PY{p}{[}\PY{l+m+mi}{1}\PY{p}{]}\PY{o}{.}\PY{n}{reshape}\PY{p}{(}\PY{l+m+mi}{2}\PY{p}{,}\PY{l+m+mi}{2}\PY{p}{)}
         \PY{k}{print} \PY{l+s}{\PYZdq{}}\PY{l+s}{Initial receiver populations state}\PY{l+s}{\PYZdq{}}
         \PY{k}{print} \PY{n}{Y\PYZus{}hist}\PY{p}{[}\PY{l+m+mi}{0}\PY{p}{]}\PY{o}{.}\PY{n}{reshape}\PY{p}{(}\PY{l+m+mi}{2}\PY{p}{,}\PY{l+m+mi}{2}\PY{p}{)}
         \PY{k}{print} \PY{l+s}{\PYZdq{}}\PY{l+s}{Next receiver populations state}\PY{l+s}{\PYZdq{}}
         \PY{k}{print} \PY{n}{Y\PYZus{}hist}\PY{p}{[}\PY{l+m+mi}{1}\PY{p}{]}\PY{o}{.}\PY{n}{reshape}\PY{p}{(}\PY{l+m+mi}{2}\PY{p}{,}\PY{l+m+mi}{2}\PY{p}{)}
\end{Verbatim}

    \begin{Verbatim}[commandchars=\\\{\}]
Initial sender populations state
[[ 0.4433709   0.5566291 ]
 [ 0.21282727  0.78717273]]
Next sender populations state
[[ 0.42059269  0.57940731]
 [ 0.26269984  0.73730016]]
Initial receiver populations state
[[ 0.71254386  0.28745614]
 [ 0.78187104  0.21812896]]
Next receiver populations state
[[ 0.83776571  0.16223429]
 [ 0.7170864   0.2829136 ]]
    \end{Verbatim}

    We should note that this is pretty quick. It takes about two seconds to
step through 10,000 steps of the simplest non-trivial signaling game.
This will obviously change as we increase the size of the matrices, but
is a good sign that everything is set up well.

    \begin{Verbatim}[commandchars=\\\{\}]
{\color{incolor}In [{\color{incolor}19}]:} \PY{k+kn}{import} \PY{n+nn}{timeit}
         \PY{n}{start} \PY{o}{=} \PY{n}{timeit}\PY{o}{.}\PY{n}{default\PYZus{}timer}\PY{p}{(}\PY{p}{)}
         \PY{n}{X\PYZus{}hist}\PY{p}{,} \PY{n}{Y\PYZus{}hist} \PY{o}{=} \PY{n}{discrete\PYZus{}time\PYZus{}replicator\PYZus{}dynamics}\PY{p}{(}\PY{l+m+mi}{10000}\PY{p}{,} \PY{n}{X}\PY{p}{,} \PY{n}{Y}\PY{p}{,} \PY{n}{A}\PY{p}{,} \PY{n}{B}\PY{p}{,} \PY{n}{P}\PY{p}{)}
         \PY{n}{stop} \PY{o}{=} \PY{n}{timeit}\PY{o}{.}\PY{n}{default\PYZus{}timer}\PY{p}{(}\PY{p}{)}
         \PY{k}{print} \PY{n}{stop} \PY{o}{\PYZhy{}} \PY{n}{start} 
\end{Verbatim}

    \begin{Verbatim}[commandchars=\\\{\}]
1.82154607773
    \end{Verbatim}

    Now that we've established that the populations are updating as they
should, we can move our attention to the long-term behavior of the
system. We'll start by running a bit longer and examining the end state.
From analytic results, we know that when states are equiprobable that
the signaling equilibria are the only asymptotically stable states. This
means that the populations should always evolve to one of two signaling
systems. We can look at this visually by plotting the proportion in the
sender and receiver populations. A signaling system results when all
sender and receiver populations converge to one or zero.

    \begin{Verbatim}[commandchars=\\\{\}]
{\color{incolor}In [{\color{incolor}20}]:} \PY{c}{\PYZsh{} Generate random starting states}
         \PY{n}{X} \PY{o}{=} \PY{n}{np}\PY{o}{.}\PY{n}{random}\PY{o}{.}\PY{n}{rand}\PY{p}{(}\PY{l+m+mi}{2}\PY{p}{,} \PY{l+m+mi}{2}\PY{p}{)}
         \PY{n}{X} \PY{o}{/}\PY{o}{=} \PY{n}{X}\PY{o}{.}\PY{n}{sum}\PY{p}{(}\PY{n}{axis}\PY{o}{=}\PY{l+m+mi}{1}\PY{p}{)}\PY{p}{[}\PY{p}{:}\PY{p}{,}\PY{n}{np}\PY{o}{.}\PY{n}{newaxis}\PY{p}{]}
         \PY{n}{Y} \PY{o}{=} \PY{n}{np}\PY{o}{.}\PY{n}{random}\PY{o}{.}\PY{n}{rand}\PY{p}{(}\PY{l+m+mi}{2}\PY{p}{,} \PY{l+m+mi}{2}\PY{p}{)}
         \PY{n}{Y} \PY{o}{/}\PY{o}{=} \PY{n}{Y}\PY{o}{.}\PY{n}{sum}\PY{p}{(}\PY{n}{axis}\PY{o}{=}\PY{l+m+mi}{1}\PY{p}{)}\PY{p}{[}\PY{p}{:}\PY{p}{,}\PY{n}{np}\PY{o}{.}\PY{n}{newaxis}\PY{p}{]}
         \PY{c}{\PYZsh{} Iterate under discrete\PYZhy{}time replicator dynamics}
         \PY{n}{X\PYZus{}hist}\PY{p}{,} \PY{n}{Y\PYZus{}hist} \PY{o}{=} \PY{n}{discrete\PYZus{}time\PYZus{}replicator\PYZus{}dynamics}\PY{p}{(}\PY{l+m+mi}{20}\PY{p}{,} \PY{n}{X}\PY{p}{,} \PY{n}{Y}\PY{p}{,} \PY{n}{A}\PY{p}{,} \PY{n}{B}\PY{p}{,} \PY{n}{P}\PY{p}{)}
         \PY{c}{\PYZsh{} Sender plots}
         \PY{n}{plt}\PY{o}{.}\PY{n}{plot}\PY{p}{(}\PY{n}{X\PYZus{}hist}\PY{p}{[}\PY{p}{:}\PY{p}{,}\PY{l+m+mi}{0}\PY{p}{]}\PY{p}{,} \PY{l+s}{\PYZsq{}}\PY{l+s}{b}\PY{l+s}{\PYZsq{}}\PY{p}{)} \PY{c}{\PYZsh{} Proportion of t\PYZus{}0 sending m\PYZus{}0}
         \PY{n}{plt}\PY{o}{.}\PY{n}{plot}\PY{p}{(}\PY{n}{X\PYZus{}hist}\PY{p}{[}\PY{p}{:}\PY{p}{,}\PY{l+m+mi}{3}\PY{p}{]}\PY{p}{,} \PY{l+s}{\PYZsq{}}\PY{l+s}{g}\PY{l+s}{\PYZsq{}}\PY{p}{)} \PY{c}{\PYZsh{} Proportion of t\PYZus{}1 sending m\PYZus{}1}
         \PY{n}{plt}\PY{o}{.}\PY{n}{ylim}\PY{p}{(}\PY{l+m+mi}{0}\PY{p}{,}\PY{l+m+mi}{1}\PY{p}{)}
         \PY{n}{plt}\PY{o}{.}\PY{n}{ylabel}\PY{p}{(}\PY{l+s}{\PYZdq{}}\PY{l+s}{Proportion of senders}\PY{l+s}{\PYZdq{}}\PY{p}{)}
         \PY{n}{plt}\PY{o}{.}\PY{n}{show}\PY{p}{(}\PY{p}{)}
         \PY{c}{\PYZsh{} Receiver plots}
         \PY{n}{plt}\PY{o}{.}\PY{n}{plot}\PY{p}{(}\PY{n}{Y\PYZus{}hist}\PY{p}{[}\PY{p}{:}\PY{p}{,}\PY{l+m+mi}{0}\PY{p}{]}\PY{p}{,} \PY{l+s}{\PYZsq{}}\PY{l+s}{b}\PY{l+s}{\PYZsq{}}\PY{p}{)} \PY{c}{\PYZsh{} Proportion of t\PYZus{}0 sending m\PYZus{}0}
         \PY{n}{plt}\PY{o}{.}\PY{n}{plot}\PY{p}{(}\PY{n}{Y\PYZus{}hist}\PY{p}{[}\PY{p}{:}\PY{p}{,}\PY{l+m+mi}{3}\PY{p}{]}\PY{p}{,} \PY{l+s}{\PYZsq{}}\PY{l+s}{g}\PY{l+s}{\PYZsq{}}\PY{p}{)} \PY{c}{\PYZsh{} Proportion of t\PYZus{}1 sending m\PYZus{}1}
         \PY{n}{plt}\PY{o}{.}\PY{n}{ylim}\PY{p}{(}\PY{l+m+mi}{0}\PY{p}{,}\PY{l+m+mi}{1}\PY{p}{)}
         \PY{n}{plt}\PY{o}{.}\PY{n}{ylabel}\PY{p}{(}\PY{l+s}{\PYZdq{}}\PY{l+s}{Proportion of receivers}\PY{l+s}{\PYZdq{}}\PY{p}{)}
         \PY{n}{plt}\PY{o}{.}\PY{n}{show}\PY{p}{(}\PY{p}{)}
\end{Verbatim}

    \begin{center}
    \adjustimage{max size={0.9\linewidth}{0.9\paperheight}}{appendixB_files/appendixB_47_0.png}
    \end{center}
    { \hspace*{\fill} \\}
    
    \begin{center}
    \adjustimage{max size={0.9\linewidth}{0.9\paperheight}}{appendixB_files/appendixB_47_1.png}
    \end{center}
    { \hspace*{\fill} \\}
    
    With simulations of signaling under the discrete-time replicator
dynamics in place, we might consider how they compare to the
continuous-time replicator dynamics.

    \subsection{Comparison with continous-time replicator
dynamics}\label{comparison-with-continous-time-replicator-dynamics}

    The continuous replicator dynamic for message $m_j$ in state $t_i$:

\begin{equation}
    \dot{x}_{ij} = x_{ij}(E[x_{ij}] - E[x_i])
\end{equation}

From above, we know that this can be expressed as:

\begin{equation}
    \dot{x}_{ij} = x_{ij}((\textbf{A}\textbf{Y}^T)_{ij} - (\textbf{X}(\textbf{A}\textbf{Y}^T)^T)_{ii})
\end{equation}

Likewise the replicator dynamic for action $a_j$ in response to message
$m_i$:

\begin{equation}
    \dot{y}_{ij} = y_{ij}(E[y_{ij}] - E[y_i])
\end{equation}

Or, alternatively:

\begin{equation}
    \dot{y}_{ij} = y_{ij}( (\textbf{B}^T\textbf{C})_{ji} - (\textbf{Y}(\textbf{B}^T\textbf{C}))_{ii})
\end{equation}

    We won't get into the details of translating the matrix notation into a
function to integrate.

    \begin{Verbatim}[commandchars=\\\{\}]
{\color{incolor}In [{\color{incolor}21}]:} \PY{k+kn}{from} \PY{n+nn}{scipy.integrate} \PY{k+kn}{import} \PY{n}{odeint}
         \PY{c}{\PYZsh{} Signaling system to integrate over}
         \PY{k}{def} \PY{n+nf}{signaling}\PY{p}{(}\PY{n}{p\PYZus{}vec}\PY{p}{,} \PY{n}{t}\PY{p}{,} \PY{n}{param}\PY{p}{)}\PY{p}{:}
             \PY{c}{\PYZsh{} Unpack the position vector}
             \PY{n}{p0}\PY{p}{,} \PY{n}{p1}\PY{p}{,} \PY{n}{q0}\PY{p}{,} \PY{n}{q1} \PY{o}{=} \PY{n}{p\PYZus{}vec}
             \PY{c}{\PYZsh{} Unpack the parametera}
             \PY{n}{a} \PY{o}{=} \PY{n}{param}\PY{p}{[}\PY{l+m+mi}{0}\PY{p}{]} \PY{c}{\PYZsh{} probability distribution P(t\PYZus{}0) = P(t\PYZus{}1) = a}
             \PY{c}{\PYZsh{} Construct system of ODEs}
             \PY{n}{p0\PYZus{}diff} \PY{o}{=} \PY{n}{p0}\PY{o}{*}\PY{p}{(}\PY{l+m+mi}{1}\PY{o}{\PYZhy{}}\PY{n}{p0}\PY{p}{)}\PY{o}{*}\PY{p}{(}\PY{n}{q0} \PY{o}{\PYZhy{}} \PY{p}{(}\PY{l+m+mi}{1} \PY{o}{\PYZhy{}} \PY{n}{q1}\PY{p}{)}\PY{p}{)}
             \PY{n}{p1\PYZus{}diff} \PY{o}{=} \PY{n}{p1}\PY{o}{*}\PY{p}{(}\PY{l+m+mi}{1}\PY{o}{\PYZhy{}}\PY{n}{p1}\PY{p}{)}\PY{o}{*}\PY{p}{(}\PY{n}{q1} \PY{o}{\PYZhy{}} \PY{p}{(}\PY{l+m+mi}{1} \PY{o}{\PYZhy{}} \PY{n}{q0}\PY{p}{)}\PY{p}{)}
             \PY{n}{q0\PYZus{}diff} \PY{o}{=} \PY{n}{q0}\PY{o}{*}\PY{p}{(}\PY{l+m+mi}{1}\PY{o}{\PYZhy{}}\PY{n}{q0}\PY{p}{)}\PY{o}{*}\PY{p}{(}\PY{n}{a}\PY{o}{*}\PY{n}{p0} \PY{o}{\PYZhy{}} \PY{p}{(}\PY{l+m+mi}{1}\PY{o}{\PYZhy{}}\PY{n}{a}\PY{p}{)}\PY{o}{*}\PY{p}{(}\PY{l+m+mi}{1}\PY{o}{\PYZhy{}}\PY{n}{p1}\PY{p}{)}\PY{p}{)}
             \PY{n}{q1\PYZus{}diff} \PY{o}{=} \PY{n}{q1}\PY{o}{*}\PY{p}{(}\PY{l+m+mi}{1}\PY{o}{\PYZhy{}}\PY{n}{q1}\PY{p}{)}\PY{o}{*}\PY{p}{(}\PY{p}{(}\PY{l+m+mi}{1}\PY{o}{\PYZhy{}}\PY{n}{a}\PY{p}{)}\PY{o}{*}\PY{n}{p1} \PY{o}{\PYZhy{}} \PY{n}{a}\PY{o}{*}\PY{p}{(}\PY{l+m+mi}{1}\PY{o}{\PYZhy{}}\PY{n}{p0}\PY{p}{)}\PY{p}{)}
             \PY{c}{\PYZsh{} Return system of ODEs}
             \PY{k}{return} \PY{p}{[}\PY{n}{p0\PYZus{}diff}\PY{p}{,} \PY{n}{p1\PYZus{}diff}\PY{p}{,} \PY{n}{q0\PYZus{}diff}\PY{p}{,} \PY{n}{q1\PYZus{}diff}\PY{p}{]}
\end{Verbatim}

    We can compare the solution trajectories of the discrete- and
continuous-time replicator dynamics. There are several things to note: *
The shape of both are generally the same, the discrete-time isn't always
``smooth'' * The discrete-time dynamic looks like a condensed version of
the continuous-time dynamics * The two kinds of dynamics sometimes yield
different solutions

    \begin{Verbatim}[commandchars=\\\{\}]
{\color{incolor}In [{\color{incolor}22}]:} \PY{c}{\PYZsh{} Create random initial states}
         \PY{n}{X} \PY{o}{=} \PY{n}{np}\PY{o}{.}\PY{n}{random}\PY{o}{.}\PY{n}{rand}\PY{p}{(}\PY{l+m+mi}{2}\PY{p}{,} \PY{l+m+mi}{2}\PY{p}{)}
         \PY{n}{X} \PY{o}{/}\PY{o}{=} \PY{n}{X}\PY{o}{.}\PY{n}{sum}\PY{p}{(}\PY{n}{axis}\PY{o}{=}\PY{l+m+mi}{1}\PY{p}{)}\PY{p}{[}\PY{p}{:}\PY{p}{,}\PY{n}{np}\PY{o}{.}\PY{n}{newaxis}\PY{p}{]}
         \PY{n}{Y} \PY{o}{=} \PY{n}{np}\PY{o}{.}\PY{n}{random}\PY{o}{.}\PY{n}{rand}\PY{p}{(}\PY{l+m+mi}{2}\PY{p}{,} \PY{l+m+mi}{2}\PY{p}{)}
         \PY{n}{Y} \PY{o}{/}\PY{o}{=} \PY{n}{Y}\PY{o}{.}\PY{n}{sum}\PY{p}{(}\PY{n}{axis}\PY{o}{=}\PY{l+m+mi}{1}\PY{p}{)}\PY{p}{[}\PY{p}{:}\PY{p}{,}\PY{n}{np}\PY{o}{.}\PY{n}{newaxis}\PY{p}{]}
         \PY{c}{\PYZsh{} Iterate discrete\PYZhy{}time replicator dynamics}
         \PY{n}{X\PYZus{}hist}\PY{p}{,} \PY{n}{Y\PYZus{}hist} \PY{o}{=} \PY{n}{discrete\PYZus{}time\PYZus{}replicator\PYZus{}dynamics}\PY{p}{(}\PY{l+m+mi}{20}\PY{p}{,} \PY{n}{X}\PY{p}{,} \PY{n}{Y}\PY{p}{,} \PY{n}{A}\PY{p}{,} \PY{n}{B}\PY{p}{,} \PY{n}{P}\PY{p}{)}
         \PY{c}{\PYZsh{} Format initial state}
         \PY{n}{p0\PYZus{}vec} \PY{o}{=} \PY{n}{X}\PY{o}{.}\PY{n}{diagonal}\PY{p}{(}\PY{p}{)}\PY{o}{.}\PY{n}{tolist}\PY{p}{(}\PY{p}{)} \PY{o}{+} \PY{n}{Y}\PY{o}{.}\PY{n}{diagonal}\PY{p}{(}\PY{p}{)}\PY{o}{.}\PY{n}{tolist}\PY{p}{(}\PY{p}{)}
         \PY{n}{params} \PY{o}{=} \PY{p}{[}\PY{n}{P}\PY{p}{[}\PY{l+m+mi}{0}\PY{p}{,}\PY{l+m+mi}{0}\PY{p}{]}\PY{p}{]}
         \PY{n}{t\PYZus{}output} \PY{o}{=} \PY{n}{np}\PY{o}{.}\PY{n}{linspace}\PY{p}{(}\PY{l+m+mi}{0}\PY{p}{,} \PY{l+m+mi}{20}\PY{p}{,} \PY{n}{num}\PY{o}{=}\PY{l+m+mi}{20}\PY{p}{)}
         \PY{c}{\PYZsh{} Integrate continuous\PYZhy{}time replicator dynamics}
         \PY{n}{p\PYZus{}vec\PYZus{}result} \PY{o}{=} \PY{n}{odeint}\PY{p}{(}\PY{n}{signaling}\PY{p}{,} \PY{n}{p0\PYZus{}vec}\PY{p}{,} \PY{n}{t\PYZus{}output}\PY{p}{,} \PY{n}{args}\PY{o}{=}\PY{p}{(}\PY{n}{params}\PY{p}{,}\PY{p}{)}\PY{p}{)}
         \PY{c}{\PYZsh{} Sender comparison}
         \PY{n}{plt}\PY{o}{.}\PY{n}{plot}\PY{p}{(}\PY{n}{X\PYZus{}hist}\PY{p}{[}\PY{p}{:}\PY{p}{,}\PY{l+m+mi}{0}\PY{p}{]}\PY{p}{,} \PY{l+s}{\PYZsq{}}\PY{l+s}{b}\PY{l+s}{\PYZsq{}}\PY{p}{)} \PY{c}{\PYZsh{} Proportion of t\PYZus{}0 sending m\PYZus{}0}
         \PY{n}{plt}\PY{o}{.}\PY{n}{plot}\PY{p}{(}\PY{n}{X\PYZus{}hist}\PY{p}{[}\PY{p}{:}\PY{p}{,}\PY{l+m+mi}{3}\PY{p}{]}\PY{p}{,} \PY{l+s}{\PYZsq{}}\PY{l+s}{g}\PY{l+s}{\PYZsq{}}\PY{p}{)} \PY{c}{\PYZsh{} Proportion of t\PYZus{}1 sending m\PYZus{}1}
         \PY{n}{plt}\PY{o}{.}\PY{n}{plot}\PY{p}{(}\PY{n}{t\PYZus{}output}\PY{p}{,} \PY{n}{p\PYZus{}vec\PYZus{}result}\PY{p}{[}\PY{p}{:}\PY{p}{,}\PY{l+m+mi}{0}\PY{p}{]}\PY{p}{,} \PY{l+s}{\PYZsq{}}\PY{l+s}{b\PYZhy{}.}\PY{l+s}{\PYZsq{}}\PY{p}{)}
         \PY{n}{plt}\PY{o}{.}\PY{n}{plot}\PY{p}{(}\PY{n}{t\PYZus{}output}\PY{p}{,} \PY{n}{p\PYZus{}vec\PYZus{}result}\PY{p}{[}\PY{p}{:}\PY{p}{,}\PY{l+m+mi}{1}\PY{p}{]}\PY{p}{,} \PY{l+s}{\PYZsq{}}\PY{l+s}{g\PYZhy{}.}\PY{l+s}{\PYZsq{}}\PY{p}{)}
         \PY{n}{plt}\PY{o}{.}\PY{n}{ylim}\PY{p}{(}\PY{l+m+mi}{0}\PY{p}{,}\PY{l+m+mi}{1}\PY{p}{)}
         \PY{n}{plt}\PY{o}{.}\PY{n}{ylabel}\PY{p}{(}\PY{l+s}{\PYZdq{}}\PY{l+s}{Proportion of senders}\PY{l+s}{\PYZdq{}}\PY{p}{)}
         \PY{n}{plt}\PY{o}{.}\PY{n}{show}\PY{p}{(}\PY{p}{)}
         \PY{c}{\PYZsh{} Receiver comparison}
         \PY{n}{plt}\PY{o}{.}\PY{n}{plot}\PY{p}{(}\PY{n}{Y\PYZus{}hist}\PY{p}{[}\PY{p}{:}\PY{p}{,}\PY{l+m+mi}{0}\PY{p}{]}\PY{p}{,} \PY{l+s}{\PYZsq{}}\PY{l+s}{b}\PY{l+s}{\PYZsq{}}\PY{p}{)} \PY{c}{\PYZsh{} Proportion of t\PYZus{}0 sending m\PYZus{}0}
         \PY{n}{plt}\PY{o}{.}\PY{n}{plot}\PY{p}{(}\PY{n}{Y\PYZus{}hist}\PY{p}{[}\PY{p}{:}\PY{p}{,}\PY{l+m+mi}{3}\PY{p}{]}\PY{p}{,} \PY{l+s}{\PYZsq{}}\PY{l+s}{g}\PY{l+s}{\PYZsq{}}\PY{p}{)} \PY{c}{\PYZsh{} Proportion of t\PYZus{}1 sending m\PYZus{}1}
         \PY{n}{plt}\PY{o}{.}\PY{n}{plot}\PY{p}{(}\PY{n}{t\PYZus{}output}\PY{p}{,} \PY{n}{p\PYZus{}vec\PYZus{}result}\PY{p}{[}\PY{p}{:}\PY{p}{,}\PY{l+m+mi}{2}\PY{p}{]}\PY{p}{,} \PY{l+s}{\PYZsq{}}\PY{l+s}{b\PYZhy{}.}\PY{l+s}{\PYZsq{}}\PY{p}{)}
         \PY{n}{plt}\PY{o}{.}\PY{n}{plot}\PY{p}{(}\PY{n}{t\PYZus{}output}\PY{p}{,} \PY{n}{p\PYZus{}vec\PYZus{}result}\PY{p}{[}\PY{p}{:}\PY{p}{,}\PY{l+m+mi}{3}\PY{p}{]}\PY{p}{,} \PY{l+s}{\PYZsq{}}\PY{l+s}{g\PYZhy{}.}\PY{l+s}{\PYZsq{}}\PY{p}{)}
         \PY{n}{plt}\PY{o}{.}\PY{n}{ylim}\PY{p}{(}\PY{l+m+mi}{0}\PY{p}{,}\PY{l+m+mi}{1}\PY{p}{)}
         \PY{n}{plt}\PY{o}{.}\PY{n}{ylabel}\PY{p}{(}\PY{l+s}{\PYZdq{}}\PY{l+s}{Proportion of receivers}\PY{l+s}{\PYZdq{}}\PY{p}{)}
         \PY{n}{plt}\PY{o}{.}\PY{n}{show}\PY{p}{(}\PY{p}{)}
\end{Verbatim}

    \begin{center}
    \adjustimage{max size={0.9\linewidth}{0.9\paperheight}}{appendixB_files/appendixB_54_0.png}
    \end{center}
    { \hspace*{\fill} \\}
    
    \begin{center}
    \adjustimage{max size={0.9\linewidth}{0.9\paperheight}}{appendixB_files/appendixB_54_1.png}
    \end{center}
    { \hspace*{\fill} \\}
    
    Let's look at a particular starting condition that yields different
solution trajectories. It looks like the scaling factors for the
discrete-time dynamics is rather large. That is, there's a really big
jump at the first iteration. This may be enough to affect the rest of
the iterations; perhaps the discretization is sufficient that it derails
the rest. One way to look at this would be to simulate random starting
points and to see when they trajectories diverge if the total movement
in the first step exceeds some threshold.

    \begin{Verbatim}[commandchars=\\\{\}]
{\color{incolor}In [{\color{incolor}23}]:} \PY{n}{X} \PY{o}{=} \PY{n}{np}\PY{o}{.}\PY{n}{array}\PY{p}{(}\PY{p}{[}\PY{p}{[} \PY{l+m+mf}{0.73162521}\PY{p}{,}  \PY{l+m+mf}{0.26837479}\PY{p}{]}\PY{p}{,} \PY{p}{[} \PY{l+m+mf}{0.27700021}\PY{p}{,}  \PY{l+m+mf}{0.72299979}\PY{p}{]}\PY{p}{]}\PY{p}{)}
         \PY{n}{Y} \PY{o}{=} \PY{n}{np}\PY{o}{.}\PY{n}{array}\PY{p}{(}\PY{p}{[}\PY{p}{[} \PY{l+m+mf}{0.58517496}\PY{p}{,}  \PY{l+m+mf}{0.41482504}\PY{p}{]}\PY{p}{,} \PY{p}{[} \PY{l+m+mf}{0.89572339}\PY{p}{,}  \PY{l+m+mf}{0.10427661}\PY{p}{]}\PY{p}{]}\PY{p}{)}
         \PY{c}{\PYZsh{}}
         \PY{n}{X\PYZus{}hist}\PY{p}{,} \PY{n}{Y\PYZus{}hist} \PY{o}{=} \PY{n}{discrete\PYZus{}time\PYZus{}replicator\PYZus{}dynamics}\PY{p}{(}\PY{l+m+mi}{20}\PY{p}{,} \PY{n}{X}\PY{p}{,} \PY{n}{Y}\PY{p}{,} \PY{n}{A}\PY{p}{,} \PY{n}{B}\PY{p}{,} \PY{n}{P}\PY{p}{)}
         \PY{c}{\PYZsh{} Format initial state}
         \PY{n}{p0\PYZus{}vec} \PY{o}{=} \PY{n}{X}\PY{o}{.}\PY{n}{diagonal}\PY{p}{(}\PY{p}{)}\PY{o}{.}\PY{n}{tolist}\PY{p}{(}\PY{p}{)} \PY{o}{+} \PY{n}{Y}\PY{o}{.}\PY{n}{diagonal}\PY{p}{(}\PY{p}{)}\PY{o}{.}\PY{n}{tolist}\PY{p}{(}\PY{p}{)}
         \PY{n}{params} \PY{o}{=} \PY{p}{[}\PY{n}{P}\PY{p}{[}\PY{l+m+mi}{0}\PY{p}{,}\PY{l+m+mi}{0}\PY{p}{]}\PY{p}{]}
         \PY{n}{t\PYZus{}output} \PY{o}{=} \PY{n}{np}\PY{o}{.}\PY{n}{linspace}\PY{p}{(}\PY{l+m+mi}{0}\PY{p}{,} \PY{l+m+mi}{20}\PY{p}{)}
         \PY{c}{\PYZsh{} Integrate continuous\PYZhy{}time replicator dynamics}
         \PY{n}{p\PYZus{}vec\PYZus{}result} \PY{o}{=} \PY{n}{odeint}\PY{p}{(}\PY{n}{signaling}\PY{p}{,} \PY{n}{p0\PYZus{}vec}\PY{p}{,} \PY{n}{t\PYZus{}output}\PY{p}{,} \PY{n}{args}\PY{o}{=}\PY{p}{(}\PY{n}{params}\PY{p}{,}\PY{p}{)}\PY{p}{)}
         \PY{c}{\PYZsh{} Sender comparison}
         \PY{n}{plt}\PY{o}{.}\PY{n}{plot}\PY{p}{(}\PY{n}{X\PYZus{}hist}\PY{p}{[}\PY{p}{:}\PY{p}{,}\PY{l+m+mi}{0}\PY{p}{]}\PY{p}{,} \PY{l+s}{\PYZsq{}}\PY{l+s}{b}\PY{l+s}{\PYZsq{}}\PY{p}{)} \PY{c}{\PYZsh{} Proportion of t\PYZus{}0 sending m\PYZus{}0}
         \PY{n}{plt}\PY{o}{.}\PY{n}{plot}\PY{p}{(}\PY{n}{X\PYZus{}hist}\PY{p}{[}\PY{p}{:}\PY{p}{,}\PY{l+m+mi}{3}\PY{p}{]}\PY{p}{,} \PY{l+s}{\PYZsq{}}\PY{l+s}{g}\PY{l+s}{\PYZsq{}}\PY{p}{)} \PY{c}{\PYZsh{} Proportion of t\PYZus{}1 sending m\PYZus{}1}
         \PY{n}{plt}\PY{o}{.}\PY{n}{plot}\PY{p}{(}\PY{n}{t\PYZus{}output}\PY{p}{,} \PY{n}{p\PYZus{}vec\PYZus{}result}\PY{p}{[}\PY{p}{:}\PY{p}{,}\PY{l+m+mi}{0}\PY{p}{]}\PY{p}{,} \PY{l+s}{\PYZsq{}}\PY{l+s}{b\PYZhy{}.}\PY{l+s}{\PYZsq{}}\PY{p}{)}
         \PY{n}{plt}\PY{o}{.}\PY{n}{plot}\PY{p}{(}\PY{n}{t\PYZus{}output}\PY{p}{,} \PY{n}{p\PYZus{}vec\PYZus{}result}\PY{p}{[}\PY{p}{:}\PY{p}{,}\PY{l+m+mi}{1}\PY{p}{]}\PY{p}{,} \PY{l+s}{\PYZsq{}}\PY{l+s}{g\PYZhy{}.}\PY{l+s}{\PYZsq{}}\PY{p}{)}
         \PY{n}{plt}\PY{o}{.}\PY{n}{ylim}\PY{p}{(}\PY{l+m+mi}{0}\PY{p}{,}\PY{l+m+mi}{1}\PY{p}{)}
         \PY{n}{plt}\PY{o}{.}\PY{n}{ylabel}\PY{p}{(}\PY{l+s}{\PYZdq{}}\PY{l+s}{Proportion of senders}\PY{l+s}{\PYZdq{}}\PY{p}{)}
         \PY{n}{plt}\PY{o}{.}\PY{n}{show}\PY{p}{(}\PY{p}{)}
         \PY{c}{\PYZsh{} Receiver comparison}
         \PY{n}{plt}\PY{o}{.}\PY{n}{plot}\PY{p}{(}\PY{n}{Y\PYZus{}hist}\PY{p}{[}\PY{p}{:}\PY{p}{,}\PY{l+m+mi}{0}\PY{p}{]}\PY{p}{,} \PY{l+s}{\PYZsq{}}\PY{l+s}{b}\PY{l+s}{\PYZsq{}}\PY{p}{)} \PY{c}{\PYZsh{} Proportion of t\PYZus{}0 sending m\PYZus{}0}
         \PY{n}{plt}\PY{o}{.}\PY{n}{plot}\PY{p}{(}\PY{n}{Y\PYZus{}hist}\PY{p}{[}\PY{p}{:}\PY{p}{,}\PY{l+m+mi}{3}\PY{p}{]}\PY{p}{,} \PY{l+s}{\PYZsq{}}\PY{l+s}{g}\PY{l+s}{\PYZsq{}}\PY{p}{)} \PY{c}{\PYZsh{} Proportion of t\PYZus{}1 sending m\PYZus{}1}
         \PY{n}{plt}\PY{o}{.}\PY{n}{plot}\PY{p}{(}\PY{n}{t\PYZus{}output}\PY{p}{,} \PY{n}{p\PYZus{}vec\PYZus{}result}\PY{p}{[}\PY{p}{:}\PY{p}{,}\PY{l+m+mi}{2}\PY{p}{]}\PY{p}{,} \PY{l+s}{\PYZsq{}}\PY{l+s}{b\PYZhy{}.}\PY{l+s}{\PYZsq{}}\PY{p}{)}
         \PY{n}{plt}\PY{o}{.}\PY{n}{plot}\PY{p}{(}\PY{n}{t\PYZus{}output}\PY{p}{,} \PY{n}{p\PYZus{}vec\PYZus{}result}\PY{p}{[}\PY{p}{:}\PY{p}{,}\PY{l+m+mi}{3}\PY{p}{]}\PY{p}{,} \PY{l+s}{\PYZsq{}}\PY{l+s}{g\PYZhy{}.}\PY{l+s}{\PYZsq{}}\PY{p}{)}
         \PY{n}{plt}\PY{o}{.}\PY{n}{ylim}\PY{p}{(}\PY{l+m+mi}{0}\PY{p}{,}\PY{l+m+mi}{1}\PY{p}{)}
         \PY{n}{plt}\PY{o}{.}\PY{n}{ylabel}\PY{p}{(}\PY{l+s}{\PYZdq{}}\PY{l+s}{Proportion of receivers}\PY{l+s}{\PYZdq{}}\PY{p}{)}
         \PY{n}{plt}\PY{o}{.}\PY{n}{show}\PY{p}{(}\PY{p}{)}
\end{Verbatim}

    \begin{center}
    \adjustimage{max size={0.9\linewidth}{0.9\paperheight}}{appendixB_files/appendixB_56_0.png}
    \end{center}
    { \hspace*{\fill} \\}
    
    \begin{center}
    \adjustimage{max size={0.9\linewidth}{0.9\paperheight}}{appendixB_files/appendixB_56_1.png}
    \end{center}
    { \hspace*{\fill} \\}
    

    % Add a bibliography block to the postdoc
    
    
    
    \end{document}
