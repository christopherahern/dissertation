This dissertation advances our understanding of the roles played by pragmatic and grammatical competence in theories of language change by applying mathematical and statistical methods to the cross-linguistic change in the expression of negation known as Jespersen?s cycle.

In the history of Middle English this change is characterized by two transitions: the first is from pre-verbal ne to an initially emphatic embracing ne...not; the second is from embracing ne...not to post-verbal not. This description conflates two often related process. The formal cycle describes changes in the forms of negation available and consists of the transitions from pre-verbal to embracing to post-verbal negation. The functional cycle de- scribes changes in how forms are used to signal meaning and consists of the transition from pre-verbal to embracing negation. While these two processes often overlap, the func- tional cycle can occur independently of the formal cycle. This informs the structure of the dissertation where we address the functional and formal cycles in turn.
Using tools from evolutionary game theory, we show that the functional cycle can be explained by limits on our pragmatic competence. In Middle English the incoming em- bracing form is initially restricted to negating propositions that are common information between interlocutors because they have recently been introduced to or can be inferred from the preceding discourse. But, experimental evidence shows that speakers have dif- ficulty in distinguishing between common and privileged information. Speakers use the initially restricted form in more and more contexts that are less and less closely tied to the discourse. As the form increases in frequency it loses the information it carried, undergoing a kind of bleaching.

Applying statistical methods developed in population genetics to a corpus of Middle English, we show that grammatical competence, and the process of acquisition through which it is formed, cannot explain either of the transitions of the formal cycle. In both cases, we would expect stability rather than change, unless the observed transitions are the result of the accumulation of small random changes akin to genetic drift in finite populations. We show that we can reject this possibility in the first transition of the formal cycle, but not in the second. The possibility of random change in the second transition of the formal cycle offers some insight into the varying amount of time it takes across languages.

The main contribution of this dissertation is demonstrating the need for articulated mod- els of both pragmatic and grammatical competence in explanatory theories of language change. It also offers a set of tools and methods for analyzing and distinguishing between different factors in quantitative historical corpus data.