\chapter{Conclusion}
\label{conclusion}

The main contribution of this dissertation is the demonstration that we need articulated models of both pragmatic and grammatical competence to construct causal models of language change. We summarize the contributions of each of the chapters towards this goal. 

Chapter 2 offered the first distinction between the formal and functional Jespersen cycles that have so often been conflated. This terminological distinction simplified the explanatory burden and made clear what there is to be explained and how we should go about the explaining. We began with the functional cycle. Chapter 3 outlined the tools from evolutionary game theory that are so well suited to modeling how information is signaled in a population of boundedly rational agents over time. Chapter 4 applied these tools to model the functional cycle, incorporating experimental evidence regarding the limits of our pragmatic competence. We then turned to the formal cycle. Chapter 5 outlined the dynamics of a model of syntactic acquisition and demonstrated that it could not account for either of the transitions of the formal cycle. Chapter 6 applied methods from population genetics to show that while  we can reject the possibility of drift in the the first transition of the formal cycle, we  cannot do so in the second transition. 

Simply put, it is our pragmatic competence that plays the central role in explaining the dynamics of change in this case. Without taking it into consideration we are left with descriptions of different states of a language without a clue as to why a transition from one to another occurred. It should be noted that this does not mean that pragmatic competence will always occupy such a role. Rather, we should be mindful of the full range of factors that influence language change.

%In Chapter 2 we began by distinguishing between two phenomena that have often been conflated in investigations of Jespersen's cycle. In particular, we argue that Jespersen's cycle as it is often described consists of both a \emph{formal} and a \emph{functional cycle}. The formal cycle describes the change in the formal complexity of negation over time. It takes place as negation becomes more and then less formally complex, as can be seen in the transitions in the history of English from \emph{\textcolor{red}{ne}} to \emph{\color{blue} ne...not}  to \emph{\color{green} not}. The functional cycle describes the way that different forms of negation are used to signal meaning. It takes place as one form of plain negation  is replaced by another form. This can be seen in the history of English from \emph{\textcolor{red}{ne}} to \emph{\color{blue} ne...not} where the originally empathic \emph{\color{blue} ne...not} displaces \emph{\textcolor{red}{ne}} as it increases in frequency, loses its emphasis, and comes to signal plain negation. We note the logical and empirical relationship between the two cycles: the functional cycle can occur independently of the formal cycle. This result informs the structure of the rest of the dissertation; we start by addressing the functional cycle before turning to the formal cycle.
%
%The first part of this dissertation addresses the functional cycle. In Chapter 3 we introduce the mathematical tools we will use to model the functional cycle. In particular, we show how we can use evolutionary game theory to describe how meaning is signaled in a population over time. Importantly, these tools allow us to model  a qualified kind of Gricean rationality. That is, individuals are \emph{boundedly rational} insofar as they have limited cognitive and informational resources \citep{simon1955,simon1957}. Yet, these tools allow us to show how the actions of individuals can give rise to change at the population level, even when those small decisions are not the product of conscious deliberation \citep{Keller:1994}. This is particularly important when we turn to the functional cycle in Chapter 4, where we show that the first transition from \emph{\textcolor{red}{ne}} to \emph{\color{blue} ne...not} can be explained as the result of speakers' limitations in keeping track of common versus private knowledge.  So, just as Gricean rationality has been used to explain particular patterns of synchronic use, a kind of bounded rationality allows us to explain the functional cycle. So, how we use these two forms explains why they change over time, and the transition from  \emph{\textcolor{red}{ne}} to \emph{\color{blue} ne...not}. 
%
%However, the same model does not apply to the transition from \emph{\color{blue} ne...not}  to \emph{\color{green} not}, so we turn to the formal cycle in the second part of this dissertation. In Chapter 5 we describe a model of syntactic acquisition and determine its predictions for both of the transitions of the formal cycle. In particular, we show that acquisition cannot explain either of the two transitions from  \emph{\textcolor{red}{ne}} to \emph{\color{blue} ne...not} or  from \emph{\color{blue} ne...not}  to \emph{\color{green} not}, other than as the result of a random change in the grammars acquired. In Chapter 6 we test this possibility using statistical methods developed in population genetics to test for random drift versus selection. We find that we can reject random drift in the case of the first transition from \emph{\textcolor{red}{ne}} to \emph{\color{blue} ne...not}, but we cannot reject drift in the case of the second drift from \emph{\color{blue} ne...not}  to \emph{\color{green} not}. This first result suggests that use is the driving force behind the first transition as part of the functional cycle. The second result shows that acquisition does not play a role in any of the observed transitions.  So, insofar as we can offer an explanation of either of the observed changes, we need the notion of pragmatic competence to do so.
