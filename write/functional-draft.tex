%\section{The Functional Cycle}
%
%If the formal cycle is defined by the forms of negation, then the functional cycle is defined by the functions that those forms are put to. It is about how and how many forms are used to mean different things. At the first stage of the cycle one form is generally used to express negation, often characterized as plain negation. Another set of forms is used to express negation in a semantically stronger sense, often characterized as being more emphatic. The first transition in the functional cycle occurs when one of the semantically stronger forms weakens to a strength intermediate between plain and emphatic negation. Thus, at the second stage of the functional cycle the available forms of negation are used to make three functional distinctions. The second transition occurs when the intermediate form weakens even further, coming to have the strength of plain negation, and the original form of plain negation is lost. In the most general sense, the functional cycle occurs when one form of plain negation is replaced by another. It is cyclic in the sense that the number of functionally distinct forms of negation increases then decreases. 
%
%This can be shown schematically as in Figure \ref{functional-cycle} where the vertical axis represents the number of functionally distinct forms of negation. The first transition takes place from \circled{2} to \circled{3} where the weakening of an emphatic form creates an additional functional distinction. The second transition takes place from \circled{3} back to \circled{2} with the loss of the distinction between the weakened emphatic form and original plain form.
%The weakening of an emphatic form increases the number of functional distinctions, and the loss of the original plain form decreases the number of distinctions. But, Figure \ref{functional-cycle} does not convey all of the necessary information about the functional cycle.  Namely, the functional cycle takes place through the change in a particular set of functional distinctions. 
%
%%Rather, the erstwhile emphatic form decreases in semantic strength and replaces the plain form.
%
%
%\begin{figure}
%\begin{tikzpicture}
%	% Define margin to offset
%	\def \margin {8}
%	% Draw nodes
%	\node[draw,circle] at ({90}:3) {3};
%	\node[draw,circle] at ({270}:3) {2};
%	% Draw arc
%	\draw[->, >=latex] ({270 - \margin}:3) arc ({270 - \margin}:{90 + \margin}:3);
%	\draw[->, >=latex] ({90 - \margin}:3) arc ({90 - \margin}:{-90 + \margin}:3);
%	% Draw complexity axis
%	\draw[->, >=latex] (-5,-3) -- (-5,3);
%	\node[align=center,text width=3cm] at (-6.75, 0) {Functional distinctions};
%\end{tikzpicture}
%\caption{The functional Jespersen cycle}
%\label{functional-cycle}
%\end{figure}
%
%
%We can more accurately represent the details of the functional cycle as  in Figure \ref{functional-cycle}, where the vertical axis represents semantic strength and the horizontal axis represents time. At the first stage of the functional cycle \circled{1} represents the original plain form, and \circled{2} represents the set of emphatic forms. At the second stage \circled{1} represents the plain form, \circled{2} the weakened emphatic form, and \circled{3} the other unaffected emphatic forms. At the third stage, \circled{2} represents the new plain form and \circled{3} the set of emphatic forms. Abstracting away from the relation between the forms, we see that there is an increase and then decrease in the number of functionally distinct forms, thus Figure \ref{functional-cycle-detail} maps onto Figure \ref{functional-cycle}.
%
%
%\begin{figure}
%\begin{center}
%\begin{tikzpicture}[->,>=stealth',shorten >=1pt,auto,node distance=3cm]
%  \node[draw,circle] (A)  {2};
%  \node[draw,circle] (B)  [right of=A] {3};
%  \node[draw,circle] (C)  [right of=B] {3};
%  \node[draw,circle] (D)  [below of=B] {2};
%  \node[draw,circle] (F)  [below of=D] {1};
%  \node[draw,circle] (E)  [left of=F] {1};
%  \node[draw,circle] (G)  [right of=F] {2};
%\path[->] (A)  edge node {} (D)
%	(B)  edge node {} (C)
%	(D)  edge node {} (G)
%	(E) edge node {} (F);
%	% Draw axes
%    \draw[->] (-1.5,-6) -- (-1.5,0);
%  \node[align=center, text width=2cm] at (-2.75, -3) {Semantic strength};
%    \draw[->] (0,-7) -- (6,-7);
%    \node at (3,-7.5) {Time};
%\end{tikzpicture}
%\end{center}
%\caption{The functional Jespersen cycle in more detail}
%\label{functional-cycle-detail}
%\end{figure}
%
%It is useful to note that the semantically stronger forms are often related to the plain form in a particular manner. That is, they are often the combination of the plain form with additional elements. 
%
%For example, at the first stage of the functional cycle in the history of English  we see plain pre-verbal \emph{\textcolor{red}{ne}}, and emphatic embracing \emph{\textcolor{blue}{ne...not}}, as in the history of French with plain pre-verbal\emph{\textcolor{red} {ne}}  and emphatic embracing \emph{\textcolor{blue}{ne...pas}}.
%
%\exg. Ic \textcolor{red}{ne} secge\\
%      I \textsc{neg} say\\
%
%\exg. I \textcolor{blue}{ne} seye \textcolor{blue}{not}\\
%      I \textsc{neg} say \textsc{neg}\\
%
%\exg. Jeo \textcolor{red}{ne} dis\\
%      I \textsc{neg} say\\
%
%\exg. Je \textcolor{blue}{ne} dis \textcolor{blue}{pas}\\
%      I \textsc{neg} say \textsc{neg}\\
%
%The initial effect of the post-verbal element, In Jespersen's words \citeyearpar[15]{jespersen:1917}:
%
%\begin{quotation}
%...[I]n most cases the addition serves to make the negative more impressive as being more vivid or picturesque, generally through an exaggeration, as when substantives meaning something very small are used as subjuncts.
%\end{quotation}
%
%Despite the evocative phrasing, Jespersen was certainly not the only or the first to notice the trajectory of the functional cycle. In fact, \cite{vanderAuwera2009} has suggested that \emph{Meillet's spiral} \citeyearpar[394]{meillet1912} may be the more appropriate term for the functional cycle.\footnote{Translation McMahon 1994, 165 : Les langues suivent ainsi une sorte de développement en spirale : elles ajoutent des mots accessoires pour obtenir une expression intense : ces mots s’affaiblissent, se dégradent et tombent au niveau de simples outils grammaticaux ; on ajoute de nouveaux mots ou des mots différents en vue de l’expression ; l’affaiblissement recommence et ainsi sans fin.} 
%\begin{quotation}
%Thus, languages follow a sort of spiral development: they add extra words to intensify expression; these words fade; decay and fall to the level of simple grammatical tools; one adds new or different words on account of expressiveness; the fading begins again, and so on endlessly.
%\end{quotation}
%
%The impact of additional elements in the history of French, among others, was noted by \citet[393]{meillet1912} for French, and even earlier by \citet[134]{gardiner1904}:
%
%\begin{quotation}
%These words, from the Latin \emph{passum} and \emph{punctum}, were originally adverbial
%accusatives placed at the end of negative sentences for the purpose of emphasis; just
%like the English ``not a jot'', ``not a straw''....\emph{Pas} and \emph{point}, and like them the
%Demotic B , Coptic AN, next lose their emphasizing force, and become mere adjuncts of
%the negative words (French \emph{ne}, Coptic 'N). Last of all, they come themselves to be looked upon as negative words.
%\end{quotation}
%
%%In this case, \emph{not} comes from Old English \emph{nawiht} (\emph{lit.} ``no thing, creature, being''), and indeed has the expected effect. The second stage in the functional cycle in English is evidenced by the use of both \emph{\textcolor{red}{ne}} and \emph{\textcolor{blue}{ne...not}}, where the second has a stronger, emphatic or exaggerative meaning. 
%
%%The meaning of the combined pre- and post-verbal elements weakens over time, and the two elements come to have the same force as the original pre-verbal element in isolation. This can be shown schematically as in Figure \ref{functional-cycle-spiral} where the different forms are arranged according to semantic strength along the vertical axis and time along the horizontal axis. 
%%
%%If we abstract away from the realization of the particular forms then Figure \ref{functional-cycle-spiral} maps onto Figure \ref{functional-cycle-detail}, which in turn maps onto Figure \ref{functional-cycle}. 
%
%%\begin{figure}
%%\begin{center}
%%\begin{tikzpicture}[->,>=stealth',shorten >=1pt,auto,node distance=3cm]
%%  \node (A)      {\textsc{\textcolor{red}{neg V}}};
%%  \node (B) [above right of=A]  {\textsc{\color{blue} neg V neg}};
%%  \node (C) [below right of=B] {\textsc{\color{blue} neg V neg}};
%%  \node (D) [below right of=A] {\textsc{\textcolor{red}{neg V}}};
%%\path[->] (A)  edge node {} (D)
%%  (B) edge node {} (C);
%%	% Draw axes
%%    \draw[->] (-1.5,-2.5) -- (-1.5,2.5);
%%  \node[align=center, text width=2cm] at (-2.75, 0) {Semantic strength};
%%    \draw[->] (0,-3) -- (4,-3);
%%    \node at (2,-3.5) {Time};
%%%  \node[align=1enter, text width=2cm] at (-2.75, 1) {Formal complexity};
%%\end{tikzpicture}
%%\end{center}
%%\caption{The realization of the functional cycle in English and French}
%%\label{functional-cycle-spiral}
%%\end{figure}
%
%
%%Also by \citet[393]{meillet1912} in French:\footnote{Translation, check with Robin: Là où l’on avait besoin d’insister sur la négation [...] on a été conduit à renforcer la négation ne ... par quelque autre mot. [...] On sait comment pas a perdu, dans les phrases où il était un accessoire de la négation, tout sons sens propre—sens conservé parfaitement dans le mot isolé pas—, comme dès lors, pas est devenu à lui seul un mot négatif, servant à exprimer la négation}
%%\begin{quotation}
%%Where we mean to insist upon negation...we are prompted to reinforce the negative \emph{ne}...with some other word....\emph{pas} itself becomes a negative word, used to express negation.
%%\end{quotation}
%
%The curious fluctuation in function that Jespersen noted becomes a particular kind of closed orbit through the space of functional distinctions, which stems from the weakening of negative forms along a semantic dimension.
%
%
%As we noted previously, the formal and functional cycle often proceed in tandem. This is the case
%
%If the first transition of the functional cycle in English and French occurs with the introduction of the optional post-verbal element, then the second transition occurs when the post-verbal element ceases to be optional. That is, when the post-vebal element becomes obligatory, it ceases to be able to carry any information beyond the fact that negation was used. The third stage of the functional cycle is a return to a single form, and thus a single function. As \cite{kiparsky-condoravdi:2006} rightly put it, to emphasize everything is to emphasize nothing.  This \emph{rhetorical devaluation} \citep{dahl:2001}, which underlies multiple linguistic phenomena, simply means that for any form, if it is the only one in use, then it cannot carry any special meaning. 
%
%%This is simply a fact about the information-theoretic capacity of signals.
%
%It bears noting that the second transition of the formal cycle does not necessarily constitute a functional cycle in its own right. That is, in the case of English and French the transition from embracing to post-verbal negation does not correspond to the same functional trajectory as the preceding transition from pre-verbal to embracing negation. Intuitively, this follows from the fact that the lexical content of post-verbal negation is a proper subset of embracing negation. If negation, like the rest of meaning is compositional, then we would not expect post-verbal negation to have a stronger or more restricted meaning than embracing negation.\footnote{The case of Brazilian Portuguese is an interesting potential exception to this rule. All three forms, pre-verbal, embracing, and post-verbal are in variation, with the post-verbal meaning having a distinct and more restricted meaning than the other two. We return to this below.}
%
%Beyond this theoretical assumption, we can also turn to historical evidence where it exists. For example, John Palsgrave, an English priest in the court of the infamous serial monogamist Henry VIII, wrote an early grammar of French entitled \emph{L'\'{e}claircissement de la langue francoyse}. The grammar was intended to help his countrymen learn French, and on the subject of negation he wrote the following helpful advice \cite[110]{palsgrave1530}.\footnote{For an electronically-available reprinting published in 1852 see: \url{https://archive.org/details/lclaircissement00wsgoog}}
%
%\begin{quotation}
%For where as they put \emph{ne} before theyr verbes, so often as they expresse negation, like as we use \emph{nat} in our tong after our verbes. They put also after theyr verbes \emph{pas}, \emph{poynt} or \emph{mye}, whiche of theym selfe signifye nothyng, but onely be as signes of negation...there is no verbe that hath \emph{ne} afore him, but he must have either \emph{pas}, \emph{point}, or \emph{mye} after hym...And note that between \emph{pas} and \emph{poynt} there is no maner difference, but it is in the speakers or writtars election whether he wyll use the one or the other. 
%\end{quotation}
%Given that Middle English exhibited both embracing and post-verbal negation at the time Palsgrave was born, the lack of distinction between the post-verbal form in English and the embracing form of French is notable. That is, he did not attribute some stronger meaning to the post-verbal form in English in relation to the embracing form. His description also points to the importance of optionality for information. Once the embracing form is obligatory it cannot carry any special meaning above and beyond that of the original pre-verbal form. 
%
%Thus, the crucial components of the functional cycle are the transition from one form which expresses a single function, to two forms that express two functions, back to a single form that necessarily expresses a single function. For the case of English and French, the functional cycle is realized  schematically as the transition from pre-verbal to embracing negation.
%
%\begin{center}
%\begin{enumerate}
%     \item \textsc{\textcolor{red}{neg V}}
%    \item  \textsc{\textcolor{red}{neg V} \textcolor{blue}{(neg)}}
%    \item \textsc{\color{blue} neg V neg}
%\end{enumerate}
%\end{center}
%This, of course, leaves out the important detail of the relation between the two forms at the second stage. Namely, the incoming form is semantically stronger. But, it demonstrates the frequent relation between the functional and formal cycles. This relationship between the formal and functional cycles is rather intuitive. Additional lexical material brings additional meaning, but only in one direction. This means that we always find a functional cycle tucked away within each formal cycle. 
%
%However, the relationship between the two cycles does not always hold. We noted above that the functional cycle can occur independently of the formal cycle. In Meillet's estimation, the functional cycle is achieved through adding new \emph{or} different words. For instance, this occurs when one form is replaced by another of equal formal complexity. \cite{kiparsky-condoravdi:2006} argue that this is exactly what takes place in the history of Greek. Historical forms of negation in Greek are listed in Table \ref{greek-table}, where emphatic negation is taken to be the semantically stronger form in comparison to \emph{plain} negation at any point in time.
%
%\begin{table}[ht]
%    \begin{center}
%    \begin{tabular}{@{}ccc@{}}
%      \hline
%      \textsc{plain} & \textsc{emphatic} & \textsc{source} \\
%      \hline
%%      ou...ti & ou-de...en & Ancient Greek \\
%      \textgreek{ou...ti} & \textgreek{ou-de...en} & Ancient Greek \\
%      \textgreek{(ou)den...ti} & \textgreek{den...tipote} & Early Medieval Greek \\
%      \textgreek{den...tipote} & \textgreek{den... prama} & Greek Dialects \\
%      \textgreek{den...prama} & \textgreek{den...apantoxh} & Modern Cretan \\
%      \hline
%    \end{tabular}
%    \end{center}
%    \caption{Historical forms of plain and emphatic negation in Greek}
%    \label{greek-table}
%\end{table}
%
%The sources of the different forms are ordered chronologically. Importantly, there is a consistent transition of forms between the two functions: the emphatic negation of the last millennium becomes the plain negation of this millennium. For example, the emphatic form in Ancient Greek, \bcode{o/u-de...en}, becomes the plain form in Early Medieval Greek, \bcode{(ou)den...ti}. This functional cycle repeats itself several times over the millennia: \bcode{ou-de...en},\bcode{den...tipote}, and \bcode{den... prama} are all replaced by a new form. The crucial point, however, is that at least some of these functional cycles occur without any concomitant formal cycle. The clearest example of this proceeds from Early Medieval Greek onwards, as is shown in Figure \ref{functional-cycle-greek}. If we were to compare the formal complexity of the forms, they would all be equivalent. All of them consist of a shared pre-verbal negator \bcode{den} along with a single post-verbal negator. We observe the functional cycle  several times in the absence of a formal cycle.
%
%%\begin{figure}
%%\begin{center}
%%\begin{tikzpicture}[->,>=stealth',shorten >=1pt,auto,node distance=3cm]
%%  \node (A)      {\textgreek{den...tipote} };
%%  \node (B) [below right of=A] {\textgreek{den...tipote} };
%%  \node (C) [above right of=A]  {\textgreek{den... prama}};
%%  \node (D) [below right of=C] {\textgreek{den... prama}};
%%  \node (E) [below right of=D] {\textgreek{den... prama}};
%%  \node (F) [above right of=D] {\textgreek{den...apantoxh}};
%%\path[->] (A)  edge node {} (B)
%%  (C) edge node {} (D)
%%  (D) edge node {} (E);
%%	% Draw axes
%%    \draw[->] (-1.5,-2.5) -- (-1.5,2.5);
%%  \node[align=center, text width=2cm] at (-2.75, 0) {Semantic strength};t
%%    \draw[->] (0,-3) -- (6,-3);
%%    \node at (3,-3.5) {Time};
%%%  \node[align=1enter, text width=2cm] at (-2.75, 1) {Formal complexity};
%%\end{tikzpicture}
%%\end{center}
%%\caption{The independent functional Jespersen cycle in Greek}
%%\label{functional-cycle-greek}
%%\end{figure}
%
%The independence of the functional cycle from the formal cycle is important, despite the fact that we will largely be concerned with its realization in English. That is, while some state of affairs may be both necessary and sufficient for the formal cycle, it can only ever be sufficient for the functional cycle. Likewise, while some state of affairs may be both necessary and sufficient for the functional cycle, it need not be necessary or sufficient for the formal cycle.  For our purposes below we will take the functional cycle to coincide with the first transition of the formal cycle. We now turn to two means of characterizing the nature of emphasis as it relates to the different forms of negation.
%

