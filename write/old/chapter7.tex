\chapter{Conclusion (15 pages)}
\label{conclusion}

Pragmatic competence shapes the input to language acquisition, language acquisition gives rise to grammatical competence, and grammatical competence determines the set of signals at the disposal of our pragmatic competence. The goal of this dissertation is to advance our understanding of the relationship between the two as repeated in Figure \ref{second}. By loosening the assumption of perfectly common interests between speakers and hearers, we can gain insight into how the use of linguistic signals changes over time, and how this impacts acquisition. By generalizing the Gricean program we have been able to situate Jespersen's Cycle at the intersection of both. Neither by itself can account for the trajectory of the cycle, but both taken together are mutually informing and revealing.

\begin{figure}
\begin{center}
\begin{tikzpicture}[->,>=stealth',shorten >=1pt,auto,node distance=3cm]
  \node (A)      {grammar $n$};
  \node (B) [below right of=A]  {data $n$};
  \node (C) [above right of=B] {grammar $n+1$};
  \node (D) [below right of=C] {data $n+1$};
  \node (E) [above right of=D] {};
\path[->] (A)  edge node {} (B)
  (B) edge node {} (C)
  (C) edge node {} (D)
  (D) edge[dashed] node {} (E);
\end{tikzpicture}
\end{center}
\caption{The iterated process of language acquisition and use}
\label{second}
\end{figure}
