This dissertation advances our understanding of the roles played by pragmatic and grammatical competence in theories of language change by using mathematical and statistical methods to analyze the cross-linguistic change in the expression of negation known as Jespersen's cycle. In the history of Middle English this change is characterized by two transitions: from pre-verbal \emph{ne} to an initially emphatic embracing \emph{ne...not}; from embracing \emph{ne...not} to post-verbal \emph{not}.  This description conflates two often related process: the formal cycle describes changes in the forms of negation available and consists of the transitions from pre-verbal to embracing to post-verbal negation; the functional cycle describes changes in how forms are used to signal meaning and consists of the transition from pre-verbal to embracing negation.

Using tools from evolutionary game theory, we show that the functional cycle can be explained by limits on our pragmatic competence.  The incoming embracing form is initially restricted to negating propositions that are common information between interlocutors. But, experimental evidence shows that speakers have difficulty in distinguishing common and privileged information. Speakers use the initially restricted form in more and more contexts that are less and less closely tied to the discourse, and it undergoes a kind of informational bleaching. Applying statistical methods developed in population genetics, we show that grammatical competence, and the process of acquisition through which it is formed, predict stability rather than change in both transitions of the formal cycle unless the observed transitions are the result of the accumulation of small random changes akin to genetic drift in finite populations. We show that we can reject this possibility in the first transition of the formal cycle, but not in the second. The possibility of random change in the second transition of the formal cycle offers some insight into the varying amount of time it takes across languages. 

The main contribution of this dissertation is demonstrating the need for articulated models of both pragmatic and grammatical competence in explanatory theories of language change, while also offering a set of tools and methods for analyzing different factors in historical corpora.